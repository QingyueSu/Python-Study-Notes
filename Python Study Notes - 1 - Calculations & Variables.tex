\documentclass[11pt]{article}

    \usepackage[breakable]{tcolorbox}
    \usepackage{parskip} % Stop auto-indenting (to mimic markdown behaviour)
    
    \usepackage{iftex}
    \ifPDFTeX
    	\usepackage[T1]{fontenc}
    	\usepackage{mathpazo}
    \else
    	\usepackage{fontspec}
    \fi

    % Basic figure setup, for now with no caption control since it's done
    % automatically by Pandoc (which extracts ![](path) syntax from Markdown).
    \usepackage{graphicx}
    % Maintain compatibility with old templates. Remove in nbconvert 6.0
    \let\Oldincludegraphics\includegraphics
    % Ensure that by default, figures have no caption (until we provide a
    % proper Figure object with a Caption API and a way to capture that
    % in the conversion process - todo).
    \usepackage{caption}
    \DeclareCaptionFormat{nocaption}{}
    \captionsetup{format=nocaption,aboveskip=0pt,belowskip=0pt}

    \usepackage[Export]{adjustbox} % Used to constrain images to a maximum size
    \adjustboxset{max size={0.9\linewidth}{0.9\paperheight}}
    \usepackage{float}
    \floatplacement{figure}{H} % forces figures to be placed at the correct location
    \usepackage{xcolor} % Allow colors to be defined
    \usepackage{enumerate} % Needed for markdown enumerations to work
    \usepackage{geometry} % Used to adjust the document margins
    \usepackage{amsmath} % Equations
    \usepackage{amssymb} % Equations
    \usepackage{textcomp} % defines textquotesingle
    % Hack from http://tex.stackexchange.com/a/47451/13684:
    \AtBeginDocument{%
        \def\PYZsq{\textquotesingle}% Upright quotes in Pygmentized code
    }
    \usepackage{upquote} % Upright quotes for verbatim code
    \usepackage{eurosym} % defines \euro
    \usepackage[mathletters]{ucs} % Extended unicode (utf-8) support
    \usepackage{fancyvrb} % verbatim replacement that allows latex
    \usepackage{grffile} % extends the file name processing of package graphics 
                         % to support a larger range
    \makeatletter % fix for grffile with XeLaTeX
    \def\Gread@@xetex#1{%
      \IfFileExists{"\Gin@base".bb}%
      {\Gread@eps{\Gin@base.bb}}%
      {\Gread@@xetex@aux#1}%
    }
    \makeatother

    % The hyperref package gives us a pdf with properly built
    % internal navigation ('pdf bookmarks' for the table of contents,
    % internal cross-reference links, web links for URLs, etc.)
    \usepackage{hyperref}
    % The default LaTeX title has an obnoxious amount of whitespace. By default,
    % titling removes some of it. It also provides customization options.
    \usepackage{titling}
    \usepackage{longtable} % longtable support required by pandoc >1.10
    \usepackage{booktabs}  % table support for pandoc > 1.12.2
    \usepackage[inline]{enumitem} % IRkernel/repr support (it uses the enumerate* environment)
    \usepackage[normalem]{ulem} % ulem is needed to support strikethroughs (\sout)
                                % normalem makes italics be italics, not underlines
    \usepackage{mathrsfs}
    

    
    % Colors for the hyperref package
    \definecolor{urlcolor}{rgb}{0,.145,.698}
    \definecolor{linkcolor}{rgb}{.71,0.21,0.01}
    \definecolor{citecolor}{rgb}{.12,.54,.11}

    % ANSI colors
    \definecolor{ansi-black}{HTML}{3E424D}
    \definecolor{ansi-black-intense}{HTML}{282C36}
    \definecolor{ansi-red}{HTML}{E75C58}
    \definecolor{ansi-red-intense}{HTML}{B22B31}
    \definecolor{ansi-green}{HTML}{00A250}
    \definecolor{ansi-green-intense}{HTML}{007427}
    \definecolor{ansi-yellow}{HTML}{DDB62B}
    \definecolor{ansi-yellow-intense}{HTML}{B27D12}
    \definecolor{ansi-blue}{HTML}{208FFB}
    \definecolor{ansi-blue-intense}{HTML}{0065CA}
    \definecolor{ansi-magenta}{HTML}{D160C4}
    \definecolor{ansi-magenta-intense}{HTML}{A03196}
    \definecolor{ansi-cyan}{HTML}{60C6C8}
    \definecolor{ansi-cyan-intense}{HTML}{258F8F}
    \definecolor{ansi-white}{HTML}{C5C1B4}
    \definecolor{ansi-white-intense}{HTML}{A1A6B2}
    \definecolor{ansi-default-inverse-fg}{HTML}{FFFFFF}
    \definecolor{ansi-default-inverse-bg}{HTML}{000000}

    % commands and environments needed by pandoc snippets
    % extracted from the output of `pandoc -s`
    \providecommand{\tightlist}{%
      \setlength{\itemsep}{0pt}\setlength{\parskip}{0pt}}
    \DefineVerbatimEnvironment{Highlighting}{Verbatim}{commandchars=\\\{\}}
    % Add ',fontsize=\small' for more characters per line
    \newenvironment{Shaded}{}{}
    \newcommand{\KeywordTok}[1]{\textcolor[rgb]{0.00,0.44,0.13}{\textbf{{#1}}}}
    \newcommand{\DataTypeTok}[1]{\textcolor[rgb]{0.56,0.13,0.00}{{#1}}}
    \newcommand{\DecValTok}[1]{\textcolor[rgb]{0.25,0.63,0.44}{{#1}}}
    \newcommand{\BaseNTok}[1]{\textcolor[rgb]{0.25,0.63,0.44}{{#1}}}
    \newcommand{\FloatTok}[1]{\textcolor[rgb]{0.25,0.63,0.44}{{#1}}}
    \newcommand{\CharTok}[1]{\textcolor[rgb]{0.25,0.44,0.63}{{#1}}}
    \newcommand{\StringTok}[1]{\textcolor[rgb]{0.25,0.44,0.63}{{#1}}}
    \newcommand{\CommentTok}[1]{\textcolor[rgb]{0.38,0.63,0.69}{\textit{{#1}}}}
    \newcommand{\OtherTok}[1]{\textcolor[rgb]{0.00,0.44,0.13}{{#1}}}
    \newcommand{\AlertTok}[1]{\textcolor[rgb]{1.00,0.00,0.00}{\textbf{{#1}}}}
    \newcommand{\FunctionTok}[1]{\textcolor[rgb]{0.02,0.16,0.49}{{#1}}}
    \newcommand{\RegionMarkerTok}[1]{{#1}}
    \newcommand{\ErrorTok}[1]{\textcolor[rgb]{1.00,0.00,0.00}{\textbf{{#1}}}}
    \newcommand{\NormalTok}[1]{{#1}}
    
    % Additional commands for more recent versions of Pandoc
    \newcommand{\ConstantTok}[1]{\textcolor[rgb]{0.53,0.00,0.00}{{#1}}}
    \newcommand{\SpecialCharTok}[1]{\textcolor[rgb]{0.25,0.44,0.63}{{#1}}}
    \newcommand{\VerbatimStringTok}[1]{\textcolor[rgb]{0.25,0.44,0.63}{{#1}}}
    \newcommand{\SpecialStringTok}[1]{\textcolor[rgb]{0.73,0.40,0.53}{{#1}}}
    \newcommand{\ImportTok}[1]{{#1}}
    \newcommand{\DocumentationTok}[1]{\textcolor[rgb]{0.73,0.13,0.13}{\textit{{#1}}}}
    \newcommand{\AnnotationTok}[1]{\textcolor[rgb]{0.38,0.63,0.69}{\textbf{\textit{{#1}}}}}
    \newcommand{\CommentVarTok}[1]{\textcolor[rgb]{0.38,0.63,0.69}{\textbf{\textit{{#1}}}}}
    \newcommand{\VariableTok}[1]{\textcolor[rgb]{0.10,0.09,0.49}{{#1}}}
    \newcommand{\ControlFlowTok}[1]{\textcolor[rgb]{0.00,0.44,0.13}{\textbf{{#1}}}}
    \newcommand{\OperatorTok}[1]{\textcolor[rgb]{0.40,0.40,0.40}{{#1}}}
    \newcommand{\BuiltInTok}[1]{{#1}}
    \newcommand{\ExtensionTok}[1]{{#1}}
    \newcommand{\PreprocessorTok}[1]{\textcolor[rgb]{0.74,0.48,0.00}{{#1}}}
    \newcommand{\AttributeTok}[1]{\textcolor[rgb]{0.49,0.56,0.16}{{#1}}}
    \newcommand{\InformationTok}[1]{\textcolor[rgb]{0.38,0.63,0.69}{\textbf{\textit{{#1}}}}}
    \newcommand{\WarningTok}[1]{\textcolor[rgb]{0.38,0.63,0.69}{\textbf{\textit{{#1}}}}}
    
    
    % Define a nice break command that doesn't care if a line doesn't already
    % exist.
    \def\br{\hspace*{\fill} \\* }
    % Math Jax compatibility definitions
    \def\gt{>}
    \def\lt{<}
    \let\Oldtex\TeX
    \let\Oldlatex\LaTeX
    \renewcommand{\TeX}{\textrm{\Oldtex}}
    \renewcommand{\LaTeX}{\textrm{\Oldlatex}}
    % Document parameters
    % Document title
    \title{Python Study Notes - 1 - Calculations \& Variables}
    
    
    
    
    
% Pygments definitions
\makeatletter
\def\PY@reset{\let\PY@it=\relax \let\PY@bf=\relax%
    \let\PY@ul=\relax \let\PY@tc=\relax%
    \let\PY@bc=\relax \let\PY@ff=\relax}
\def\PY@tok#1{\csname PY@tok@#1\endcsname}
\def\PY@toks#1+{\ifx\relax#1\empty\else%
    \PY@tok{#1}\expandafter\PY@toks\fi}
\def\PY@do#1{\PY@bc{\PY@tc{\PY@ul{%
    \PY@it{\PY@bf{\PY@ff{#1}}}}}}}
\def\PY#1#2{\PY@reset\PY@toks#1+\relax+\PY@do{#2}}

\expandafter\def\csname PY@tok@w\endcsname{\def\PY@tc##1{\textcolor[rgb]{0.73,0.73,0.73}{##1}}}
\expandafter\def\csname PY@tok@c\endcsname{\let\PY@it=\textit\def\PY@tc##1{\textcolor[rgb]{0.25,0.50,0.50}{##1}}}
\expandafter\def\csname PY@tok@cp\endcsname{\def\PY@tc##1{\textcolor[rgb]{0.74,0.48,0.00}{##1}}}
\expandafter\def\csname PY@tok@k\endcsname{\let\PY@bf=\textbf\def\PY@tc##1{\textcolor[rgb]{0.00,0.50,0.00}{##1}}}
\expandafter\def\csname PY@tok@kp\endcsname{\def\PY@tc##1{\textcolor[rgb]{0.00,0.50,0.00}{##1}}}
\expandafter\def\csname PY@tok@kt\endcsname{\def\PY@tc##1{\textcolor[rgb]{0.69,0.00,0.25}{##1}}}
\expandafter\def\csname PY@tok@o\endcsname{\def\PY@tc##1{\textcolor[rgb]{0.40,0.40,0.40}{##1}}}
\expandafter\def\csname PY@tok@ow\endcsname{\let\PY@bf=\textbf\def\PY@tc##1{\textcolor[rgb]{0.67,0.13,1.00}{##1}}}
\expandafter\def\csname PY@tok@nb\endcsname{\def\PY@tc##1{\textcolor[rgb]{0.00,0.50,0.00}{##1}}}
\expandafter\def\csname PY@tok@nf\endcsname{\def\PY@tc##1{\textcolor[rgb]{0.00,0.00,1.00}{##1}}}
\expandafter\def\csname PY@tok@nc\endcsname{\let\PY@bf=\textbf\def\PY@tc##1{\textcolor[rgb]{0.00,0.00,1.00}{##1}}}
\expandafter\def\csname PY@tok@nn\endcsname{\let\PY@bf=\textbf\def\PY@tc##1{\textcolor[rgb]{0.00,0.00,1.00}{##1}}}
\expandafter\def\csname PY@tok@ne\endcsname{\let\PY@bf=\textbf\def\PY@tc##1{\textcolor[rgb]{0.82,0.25,0.23}{##1}}}
\expandafter\def\csname PY@tok@nv\endcsname{\def\PY@tc##1{\textcolor[rgb]{0.10,0.09,0.49}{##1}}}
\expandafter\def\csname PY@tok@no\endcsname{\def\PY@tc##1{\textcolor[rgb]{0.53,0.00,0.00}{##1}}}
\expandafter\def\csname PY@tok@nl\endcsname{\def\PY@tc##1{\textcolor[rgb]{0.63,0.63,0.00}{##1}}}
\expandafter\def\csname PY@tok@ni\endcsname{\let\PY@bf=\textbf\def\PY@tc##1{\textcolor[rgb]{0.60,0.60,0.60}{##1}}}
\expandafter\def\csname PY@tok@na\endcsname{\def\PY@tc##1{\textcolor[rgb]{0.49,0.56,0.16}{##1}}}
\expandafter\def\csname PY@tok@nt\endcsname{\let\PY@bf=\textbf\def\PY@tc##1{\textcolor[rgb]{0.00,0.50,0.00}{##1}}}
\expandafter\def\csname PY@tok@nd\endcsname{\def\PY@tc##1{\textcolor[rgb]{0.67,0.13,1.00}{##1}}}
\expandafter\def\csname PY@tok@s\endcsname{\def\PY@tc##1{\textcolor[rgb]{0.73,0.13,0.13}{##1}}}
\expandafter\def\csname PY@tok@sd\endcsname{\let\PY@it=\textit\def\PY@tc##1{\textcolor[rgb]{0.73,0.13,0.13}{##1}}}
\expandafter\def\csname PY@tok@si\endcsname{\let\PY@bf=\textbf\def\PY@tc##1{\textcolor[rgb]{0.73,0.40,0.53}{##1}}}
\expandafter\def\csname PY@tok@se\endcsname{\let\PY@bf=\textbf\def\PY@tc##1{\textcolor[rgb]{0.73,0.40,0.13}{##1}}}
\expandafter\def\csname PY@tok@sr\endcsname{\def\PY@tc##1{\textcolor[rgb]{0.73,0.40,0.53}{##1}}}
\expandafter\def\csname PY@tok@ss\endcsname{\def\PY@tc##1{\textcolor[rgb]{0.10,0.09,0.49}{##1}}}
\expandafter\def\csname PY@tok@sx\endcsname{\def\PY@tc##1{\textcolor[rgb]{0.00,0.50,0.00}{##1}}}
\expandafter\def\csname PY@tok@m\endcsname{\def\PY@tc##1{\textcolor[rgb]{0.40,0.40,0.40}{##1}}}
\expandafter\def\csname PY@tok@gh\endcsname{\let\PY@bf=\textbf\def\PY@tc##1{\textcolor[rgb]{0.00,0.00,0.50}{##1}}}
\expandafter\def\csname PY@tok@gu\endcsname{\let\PY@bf=\textbf\def\PY@tc##1{\textcolor[rgb]{0.50,0.00,0.50}{##1}}}
\expandafter\def\csname PY@tok@gd\endcsname{\def\PY@tc##1{\textcolor[rgb]{0.63,0.00,0.00}{##1}}}
\expandafter\def\csname PY@tok@gi\endcsname{\def\PY@tc##1{\textcolor[rgb]{0.00,0.63,0.00}{##1}}}
\expandafter\def\csname PY@tok@gr\endcsname{\def\PY@tc##1{\textcolor[rgb]{1.00,0.00,0.00}{##1}}}
\expandafter\def\csname PY@tok@ge\endcsname{\let\PY@it=\textit}
\expandafter\def\csname PY@tok@gs\endcsname{\let\PY@bf=\textbf}
\expandafter\def\csname PY@tok@gp\endcsname{\let\PY@bf=\textbf\def\PY@tc##1{\textcolor[rgb]{0.00,0.00,0.50}{##1}}}
\expandafter\def\csname PY@tok@go\endcsname{\def\PY@tc##1{\textcolor[rgb]{0.53,0.53,0.53}{##1}}}
\expandafter\def\csname PY@tok@gt\endcsname{\def\PY@tc##1{\textcolor[rgb]{0.00,0.27,0.87}{##1}}}
\expandafter\def\csname PY@tok@err\endcsname{\def\PY@bc##1{\setlength{\fboxsep}{0pt}\fcolorbox[rgb]{1.00,0.00,0.00}{1,1,1}{\strut ##1}}}
\expandafter\def\csname PY@tok@kc\endcsname{\let\PY@bf=\textbf\def\PY@tc##1{\textcolor[rgb]{0.00,0.50,0.00}{##1}}}
\expandafter\def\csname PY@tok@kd\endcsname{\let\PY@bf=\textbf\def\PY@tc##1{\textcolor[rgb]{0.00,0.50,0.00}{##1}}}
\expandafter\def\csname PY@tok@kn\endcsname{\let\PY@bf=\textbf\def\PY@tc##1{\textcolor[rgb]{0.00,0.50,0.00}{##1}}}
\expandafter\def\csname PY@tok@kr\endcsname{\let\PY@bf=\textbf\def\PY@tc##1{\textcolor[rgb]{0.00,0.50,0.00}{##1}}}
\expandafter\def\csname PY@tok@bp\endcsname{\def\PY@tc##1{\textcolor[rgb]{0.00,0.50,0.00}{##1}}}
\expandafter\def\csname PY@tok@fm\endcsname{\def\PY@tc##1{\textcolor[rgb]{0.00,0.00,1.00}{##1}}}
\expandafter\def\csname PY@tok@vc\endcsname{\def\PY@tc##1{\textcolor[rgb]{0.10,0.09,0.49}{##1}}}
\expandafter\def\csname PY@tok@vg\endcsname{\def\PY@tc##1{\textcolor[rgb]{0.10,0.09,0.49}{##1}}}
\expandafter\def\csname PY@tok@vi\endcsname{\def\PY@tc##1{\textcolor[rgb]{0.10,0.09,0.49}{##1}}}
\expandafter\def\csname PY@tok@vm\endcsname{\def\PY@tc##1{\textcolor[rgb]{0.10,0.09,0.49}{##1}}}
\expandafter\def\csname PY@tok@sa\endcsname{\def\PY@tc##1{\textcolor[rgb]{0.73,0.13,0.13}{##1}}}
\expandafter\def\csname PY@tok@sb\endcsname{\def\PY@tc##1{\textcolor[rgb]{0.73,0.13,0.13}{##1}}}
\expandafter\def\csname PY@tok@sc\endcsname{\def\PY@tc##1{\textcolor[rgb]{0.73,0.13,0.13}{##1}}}
\expandafter\def\csname PY@tok@dl\endcsname{\def\PY@tc##1{\textcolor[rgb]{0.73,0.13,0.13}{##1}}}
\expandafter\def\csname PY@tok@s2\endcsname{\def\PY@tc##1{\textcolor[rgb]{0.73,0.13,0.13}{##1}}}
\expandafter\def\csname PY@tok@sh\endcsname{\def\PY@tc##1{\textcolor[rgb]{0.73,0.13,0.13}{##1}}}
\expandafter\def\csname PY@tok@s1\endcsname{\def\PY@tc##1{\textcolor[rgb]{0.73,0.13,0.13}{##1}}}
\expandafter\def\csname PY@tok@mb\endcsname{\def\PY@tc##1{\textcolor[rgb]{0.40,0.40,0.40}{##1}}}
\expandafter\def\csname PY@tok@mf\endcsname{\def\PY@tc##1{\textcolor[rgb]{0.40,0.40,0.40}{##1}}}
\expandafter\def\csname PY@tok@mh\endcsname{\def\PY@tc##1{\textcolor[rgb]{0.40,0.40,0.40}{##1}}}
\expandafter\def\csname PY@tok@mi\endcsname{\def\PY@tc##1{\textcolor[rgb]{0.40,0.40,0.40}{##1}}}
\expandafter\def\csname PY@tok@il\endcsname{\def\PY@tc##1{\textcolor[rgb]{0.40,0.40,0.40}{##1}}}
\expandafter\def\csname PY@tok@mo\endcsname{\def\PY@tc##1{\textcolor[rgb]{0.40,0.40,0.40}{##1}}}
\expandafter\def\csname PY@tok@ch\endcsname{\let\PY@it=\textit\def\PY@tc##1{\textcolor[rgb]{0.25,0.50,0.50}{##1}}}
\expandafter\def\csname PY@tok@cm\endcsname{\let\PY@it=\textit\def\PY@tc##1{\textcolor[rgb]{0.25,0.50,0.50}{##1}}}
\expandafter\def\csname PY@tok@cpf\endcsname{\let\PY@it=\textit\def\PY@tc##1{\textcolor[rgb]{0.25,0.50,0.50}{##1}}}
\expandafter\def\csname PY@tok@c1\endcsname{\let\PY@it=\textit\def\PY@tc##1{\textcolor[rgb]{0.25,0.50,0.50}{##1}}}
\expandafter\def\csname PY@tok@cs\endcsname{\let\PY@it=\textit\def\PY@tc##1{\textcolor[rgb]{0.25,0.50,0.50}{##1}}}

\def\PYZbs{\char`\\}
\def\PYZus{\char`\_}
\def\PYZob{\char`\{}
\def\PYZcb{\char`\}}
\def\PYZca{\char`\^}
\def\PYZam{\char`\&}
\def\PYZlt{\char`\<}
\def\PYZgt{\char`\>}
\def\PYZsh{\char`\#}
\def\PYZpc{\char`\%}
\def\PYZdl{\char`\$}
\def\PYZhy{\char`\-}
\def\PYZsq{\char`\'}
\def\PYZdq{\char`\"}
\def\PYZti{\char`\~}
% for compatibility with earlier versions
\def\PYZat{@}
\def\PYZlb{[}
\def\PYZrb{]}
\makeatother


    % For linebreaks inside Verbatim environment from package fancyvrb. 
    \makeatletter
        \newbox\Wrappedcontinuationbox 
        \newbox\Wrappedvisiblespacebox 
        \newcommand*\Wrappedvisiblespace {\textcolor{red}{\textvisiblespace}} 
        \newcommand*\Wrappedcontinuationsymbol {\textcolor{red}{\llap{\tiny$\m@th\hookrightarrow$}}} 
        \newcommand*\Wrappedcontinuationindent {3ex } 
        \newcommand*\Wrappedafterbreak {\kern\Wrappedcontinuationindent\copy\Wrappedcontinuationbox} 
        % Take advantage of the already applied Pygments mark-up to insert 
        % potential linebreaks for TeX processing. 
        %        {, <, #, %, $, ' and ": go to next line. 
        %        _, }, ^, &, >, - and ~: stay at end of broken line. 
        % Use of \textquotesingle for straight quote. 
        \newcommand*\Wrappedbreaksatspecials {% 
            \def\PYGZus{\discretionary{\char`\_}{\Wrappedafterbreak}{\char`\_}}% 
            \def\PYGZob{\discretionary{}{\Wrappedafterbreak\char`\{}{\char`\{}}% 
            \def\PYGZcb{\discretionary{\char`\}}{\Wrappedafterbreak}{\char`\}}}% 
            \def\PYGZca{\discretionary{\char`\^}{\Wrappedafterbreak}{\char`\^}}% 
            \def\PYGZam{\discretionary{\char`\&}{\Wrappedafterbreak}{\char`\&}}% 
            \def\PYGZlt{\discretionary{}{\Wrappedafterbreak\char`\<}{\char`\<}}% 
            \def\PYGZgt{\discretionary{\char`\>}{\Wrappedafterbreak}{\char`\>}}% 
            \def\PYGZsh{\discretionary{}{\Wrappedafterbreak\char`\#}{\char`\#}}% 
            \def\PYGZpc{\discretionary{}{\Wrappedafterbreak\char`\%}{\char`\%}}% 
            \def\PYGZdl{\discretionary{}{\Wrappedafterbreak\char`\$}{\char`\$}}% 
            \def\PYGZhy{\discretionary{\char`\-}{\Wrappedafterbreak}{\char`\-}}% 
            \def\PYGZsq{\discretionary{}{\Wrappedafterbreak\textquotesingle}{\textquotesingle}}% 
            \def\PYGZdq{\discretionary{}{\Wrappedafterbreak\char`\"}{\char`\"}}% 
            \def\PYGZti{\discretionary{\char`\~}{\Wrappedafterbreak}{\char`\~}}% 
        } 
        % Some characters . , ; ? ! / are not pygmentized. 
        % This macro makes them "active" and they will insert potential linebreaks 
        \newcommand*\Wrappedbreaksatpunct {% 
            \lccode`\~`\.\lowercase{\def~}{\discretionary{\hbox{\char`\.}}{\Wrappedafterbreak}{\hbox{\char`\.}}}% 
            \lccode`\~`\,\lowercase{\def~}{\discretionary{\hbox{\char`\,}}{\Wrappedafterbreak}{\hbox{\char`\,}}}% 
            \lccode`\~`\;\lowercase{\def~}{\discretionary{\hbox{\char`\;}}{\Wrappedafterbreak}{\hbox{\char`\;}}}% 
            \lccode`\~`\:\lowercase{\def~}{\discretionary{\hbox{\char`\:}}{\Wrappedafterbreak}{\hbox{\char`\:}}}% 
            \lccode`\~`\?\lowercase{\def~}{\discretionary{\hbox{\char`\?}}{\Wrappedafterbreak}{\hbox{\char`\?}}}% 
            \lccode`\~`\!\lowercase{\def~}{\discretionary{\hbox{\char`\!}}{\Wrappedafterbreak}{\hbox{\char`\!}}}% 
            \lccode`\~`\/\lowercase{\def~}{\discretionary{\hbox{\char`\/}}{\Wrappedafterbreak}{\hbox{\char`\/}}}% 
            \catcode`\.\active
            \catcode`\,\active 
            \catcode`\;\active
            \catcode`\:\active
            \catcode`\?\active
            \catcode`\!\active
            \catcode`\/\active 
            \lccode`\~`\~ 	
        }
    \makeatother

    \let\OriginalVerbatim=\Verbatim
    \makeatletter
    \renewcommand{\Verbatim}[1][1]{%
        %\parskip\z@skip
        \sbox\Wrappedcontinuationbox {\Wrappedcontinuationsymbol}%
        \sbox\Wrappedvisiblespacebox {\FV@SetupFont\Wrappedvisiblespace}%
        \def\FancyVerbFormatLine ##1{\hsize\linewidth
            \vtop{\raggedright\hyphenpenalty\z@\exhyphenpenalty\z@
                \doublehyphendemerits\z@\finalhyphendemerits\z@
                \strut ##1\strut}%
        }%
        % If the linebreak is at a space, the latter will be displayed as visible
        % space at end of first line, and a continuation symbol starts next line.
        % Stretch/shrink are however usually zero for typewriter font.
        \def\FV@Space {%
            \nobreak\hskip\z@ plus\fontdimen3\font minus\fontdimen4\font
            \discretionary{\copy\Wrappedvisiblespacebox}{\Wrappedafterbreak}
            {\kern\fontdimen2\font}%
        }%
        
        % Allow breaks at special characters using \PYG... macros.
        \Wrappedbreaksatspecials
        % Breaks at punctuation characters . , ; ? ! and / need catcode=\active 	
        \OriginalVerbatim[#1,codes*=\Wrappedbreaksatpunct]%
    }
    \makeatother

    % Exact colors from NB
    \definecolor{incolor}{HTML}{303F9F}
    \definecolor{outcolor}{HTML}{D84315}
    \definecolor{cellborder}{HTML}{CFCFCF}
    \definecolor{cellbackground}{HTML}{F7F7F7}
    
    % prompt
    \makeatletter
    \newcommand{\boxspacing}{\kern\kvtcb@left@rule\kern\kvtcb@boxsep}
    \makeatother
    \newcommand{\prompt}[4]{
        \ttfamily\llap{{\color{#2}[#3]:\hspace{3pt}#4}}\vspace{-\baselineskip}
    }
    

    
    % Prevent overflowing lines due to hard-to-break entities
    \sloppy 
    % Setup hyperref package
    \hypersetup{
      breaklinks=true,  % so long urls are correctly broken across lines
      colorlinks=true,
      urlcolor=urlcolor,
      linkcolor=linkcolor,
      citecolor=citecolor,
      }
    % Slightly bigger margins than the latex defaults
    
    \geometry{verbose,tmargin=1in,bmargin=1in,lmargin=1in,rmargin=1in}
    
    

\begin{document}
    
    \maketitle
    
    

    
    Python Study Notes - 1

by Sue

    README

In this series of notebooks, I will summarize all the information I find
useful when I was studying Python. I will display the contents from the
most basic part to the most difficult part according to my
understanding.

This notebook is the first one, which is the basic introduction of some
calculations and basic usages of Python.

    Table of contents

\begin{verbatim}
<ul>
    <li><a id="#Part0">Part 0. Python background</a></li>
        <ul>
            <li><a id="#Part0_1">1. Basic usage of common signs</a></li>
                <ul>
                    <li><a id="#Part0_1_1">(1) #</a></li>
                    <li><a id="#Part0_1_2">(2) " "&' '&''' '''</a></li>
                    <li><a id="#Part0_1_3">(3) ;</a></li>
                    <li><a id="#Part0_1_4">(4) \</a></li>
                </ul>
                <li><a id="#Part0_2">2. Basic functions</a></li>
                <ul>
                    <li><a id="#Part0_2_1">(1) format()</a></li>
                    <li><a id="#Part0_2_2">(2) print()</a></li>
                    <li><a id="#Part0_2_3">(3) range()</a></li>
                    <li><a id="#Part0_2_4">(4) id()</a></li>
                    <li><a id="#Part0_2_5">(5) functions in random package</a></li>
                </ul>
            <li><a id="#Part0_3">3. Basic regulations about naming variables</a></li>
        </ul>
    <li><a id="#Part1">Part 1. Python as a calculator</a></li>
        <ul>
            <li><a id="#Part1_1">1. Basic types of calculation</a></li>
                <ul>
                    <li><a id="#Part1_1_1">(1) Addition: +</a></li>
                    <li><a id="#Part1_1_2">(2) Subtraction: -</a></li>
                    <li><a id="#Part1_1_3">(3) Multiplication: *</a></li>
                    <li><a id="#Part1_1_4">(4) Division: /</a></li>
                    <li><a id="#Part1_1_5">(5) Exponentiation: **</a></li>
                    <li><a id="#Part1_1_6">(6) Round down: //</a></li>
                    <li><a id="#Part1_1_7">(7) Modulo: %</a></li>
                </ul>
            <li><a id="#Part1_2">2. Special types of calculation (relevant to decimal system and Binary system)</a></li>
                <ul>
                    <li><a id="#Part1_2_1">(1) Left move: <<</a></li>
                    <li><a id="#Part1_2_2">(2) Right move: >></a></li>
                    <li><a id="#Part1_2_3">(3) Bitwise AND: &</a></li>
                    <li><a id="#Part1_2_4">(4) Bitwise OR: |</a></li>
                    <li><a id="#Part1_2_5">(5) Bitwise XOR: ^</a></li>
                    <li><a id="#Part1_2_6">(6) compl: ~</a></li>
                </ul>
            <li><a id="#Part1_3">3. Special types of calculation (relevant to boolean variable)</a></li>
                <ul>
                    <li><a id="#Part1_3_1">(1) Smaller than: <</a></li>
                    <li><a id="#Part1_3_2">(2) Larger than: ></a></li>
                    <li><a id="#Part1_3_3">(3) Smaller than or equal to: <=</a></li>
                    <li><a id="#Part1_3_4">(4) Larger than or equal to: >=</a></li>
                    <li><a id="#Part1_3_5">(5) Equal to: ==</a></li>
                    <li><a id="#Part1_3_6">(6) Not equal to: !=</a></li>
                    <li><a id="#Part1_3_7">(7) Boolean Not: not</a></li>
                    <li><a id="#Part1_3_8">(8) Boolean And: and</a></li>
                    <li><a id="#Part1_3_9">(9) Boolean Or: or</a></li>
                </ul>
        </ul>
    <li><a id="#Part2">Part 2. Variables and Types</a></li>
        <ul>
            <li><a id="#Part2_1">1. Five different types of variables</a></li>
                <ul>
                    <li><a id="#Part2_1_1">(1) integer: int</a></li>
                    <li><a id="#Part2_1_2">(2) float: float (special: long, None)</a></li>
                    <li><a id="#Part2_1_3">(3) string: str (special: unicode)</a></li>
                    <li><a id="#Part2_1_4">(4) boolean: bool</a></li>
                    <li><a id="#Part2_1_5">(5) date</a></li>
                </ul>
            <li><a id="#Part2_2">2. Calculation among variables</a></li>
                <ul>
                    <li><a id="#Part2_2_1">(1) among numericals (integer or float)</a></li>
                    <li><a id="#Part2_2_2">(2) among strings</a></li>
                </ul>
            <li><a id="#Part2_3">3. Transformation between different types of variables</a></li>
                <ul>
                    <li><a id="#Part2_3_1">(1) str()</a></li>
                    <li><a id="#Part2_3_2">(2) int() / float()</a></li>
                </ul>
        </ul>
    <li><a id="#Part3">Part 3. Control Flow Statement</a></li>   
        <ul>
            <li><a id="#Part3_1">1. three types of control flow statements</a></li>
                <ul>
                    <li><a id="#Part3_1_1">(1) 'if' statement</a></li>
                    <li><a id="#Part3_1_2">(2) 'while' statement</a></li>
                    <li><a id="#Part3_1_3">(3) 'for' statement</a></li>
                </ul>
            <li><a id="#Part3_2">2. Three types of statements used in the three types of control flow statements above</a></li>
                <ul>
                    <li><a id="#Part3_2_1">(1) 'pass' statement</a></li>
                    <li><a id="#Part3_2_2">(2) 'continue' statement</a></li>
                    <li><a id="#Part3_2_3">(3) 'break' statement</a></li>
                    <li><a id="#Part3_2_4">(4) 'return' statement (*)</a></li>
                </ul>
        </ul>
    <li><a id="#Part4">Part 4. Functions</a></li>   
        <ul>
            <li><a id="#Part4_1">1. Basic Approach to use function</a></li>
                <ul>
                    <li><a id="#Part4_1_1">(1) function without parameters</a></li>
                    <li><a id="#Part4_1_2">(2) function with parameters</a></li>
                </ul>
            <li><a id="#Part4_2">2. Local Variables & Global Variables</a></li>
                <ul>
                    <li><a id="#Part4_2_1">(1) Local Variables</a></li>
                    <li><a id="#Part4_2_2">(2) Global Variables</a></li>
                </ul>
            <li><a id="#Part4_3">3. Default value  & Keyword Arguments</a></li>
                <ul>
                    <li><a id="#Part4_3_1">(1) Default value</a></li>
                    <li><a id="#Part4_3_2">(2) Keyword Arguments</a></li>
                </ul>
            <li><a id="#Part4_4">4. Changeable Parameters</a></li>
            <li><a id="#Part4_5">5. Documentation Strings (DocStrings)</a></li>
        </ul>
    <li><a id="#Part5">Part 5. Modules & Packages</a></li>   
        <ul>
            <li><a id="#Part5_1">1. Modules</a></li>
        </ul>
</ul>
\end{verbatim}

    Part 0. Python background

    \begin{enumerate}
\def\labelenumi{\arabic{enumi}.}
\tightlist
\item
  Basic usage of common signs
\end{enumerate}

    \begin{enumerate}
\def\labelenumi{(\arabic{enumi})}
\tightlist
\item
  \#
\end{enumerate}

    \begin{itemize}
\tightlist
\item
  \# : comments
\end{itemize}

    \begin{enumerate}
\def\labelenumi{(\arabic{enumi})}
\setcounter{enumi}{1}
\tightlist
\item
  " "\&' `\&''\,' '\,'\,'
\end{enumerate}

    \begin{itemize}
\tightlist
\item
  ' ' or " " :

  \begin{itemize}
  \tightlist
  \item
    both refer the specific string (one line)
  \item
    {[}But '\,' can not be referred, "" can be.{]}
  \end{itemize}
\item
  '\,'`\ldots{}''\,' :

  \begin{itemize}
  \tightlist
  \item
    refer the specific string (more than one lines)
  \item
    {[}You can use '\,' or "" in this string part{]}
  \end{itemize}
\end{itemize}

    \begin{enumerate}
\def\labelenumi{(\arabic{enumi})}
\setcounter{enumi}{2}
\tightlist
\item
  ;
\end{enumerate}

    \begin{itemize}
\tightlist
\item
  ; : use to make more than on commands on the one line
\end{itemize}

    \begin{tcolorbox}[breakable, size=fbox, boxrule=1pt, pad at break*=1mm,colback=cellbackground, colframe=cellborder]
\prompt{In}{incolor}{71}{\boxspacing}
\begin{Verbatim}[commandchars=\\\{\}]
\PY{n}{i} \PY{o}{=} \PY{l+m+mi}{5}
\PY{n+nb}{print}\PY{p}{(}\PY{n}{i}\PY{p}{)}
\PY{c+c1}{\PYZsh{} which is the same as follows:}
\PY{n}{i} \PY{o}{=} \PY{l+m+mi}{5}\PY{p}{;} \PY{n+nb}{print}\PY{p}{(}\PY{n}{i}\PY{p}{)}
\end{Verbatim}
\end{tcolorbox}

    \begin{Verbatim}[commandchars=\\\{\}]
5
5
    \end{Verbatim}

    \begin{enumerate}
\def\labelenumi{(\arabic{enumi})}
\setcounter{enumi}{3}
\tightlist
\item
  \textless/h4\textgreater{}
\end{enumerate}

    \begin{itemize}
\item
  \textbackslash: It has two meanings, one is escape sequence, and
  another is raw strings.

  \begin{itemize}
  \tightlist
  \item
    escape sequence

    \begin{itemize}
    \item
      \begin{enumerate}
      \def\labelenumi{\arabic{enumi})}
      \tightlist
      \item
        \n: represents a new line
      \end{enumerate}
    \item
      \begin{enumerate}
      \def\labelenumi{\arabic{enumi})}
      \setcounter{enumi}{1}
      \tightlist
      \item
        \t: (??? what's this?)
      \end{enumerate}
    \item
      \begin{enumerate}
      \def\labelenumi{\arabic{enumi})}
      \setcounter{enumi}{2}
      \tightlist
      \item
        \textbackslash: make the original multiple lines to show in one
        line (Sometimes, the code was supposed to be finished in one
        line, but it is too much to be in one line, so we have to show
        it in multiple lines or it is better to show in multiple lines.
        In this case, we need to use this sign.
      \end{enumerate}
    \end{itemize}
  \item
    raw strings
  \end{itemize}
\end{itemize}

    \begin{tcolorbox}[breakable, size=fbox, boxrule=1pt, pad at break*=1mm,colback=cellbackground, colframe=cellborder]
\prompt{In}{incolor}{72}{\boxspacing}
\begin{Verbatim}[commandchars=\\\{\}]
\PY{c+c1}{\PYZsh{} escape sequence \PYZbs{}n}
\PY{n+nb}{print}\PY{p}{(}\PY{l+s+s1}{\PYZsq{}}\PY{l+s+s1}{This is the first line}\PY{l+s+se}{\PYZbs{}n}\PY{l+s+s1}{This is the second line}\PY{l+s+s1}{\PYZsq{}}\PY{p}{)}
\end{Verbatim}
\end{tcolorbox}

    \begin{Verbatim}[commandchars=\\\{\}]
This is the first line
This is the second line
    \end{Verbatim}

    \begin{tcolorbox}[breakable, size=fbox, boxrule=1pt, pad at break*=1mm,colback=cellbackground, colframe=cellborder]
\prompt{In}{incolor}{73}{\boxspacing}
\begin{Verbatim}[commandchars=\\\{\}]
\PY{c+c1}{\PYZsh{} escape sequence \PYZbs{}}
\PY{n+nb}{print}\PY{p}{(}\PY{l+s+s2}{\PYZdq{}}\PY{l+s+s2}{This is the first line. }\PY{l+s+se}{\PYZbs{}}
\PY{l+s+s2}{This is the second line.}\PY{l+s+s2}{\PYZdq{}}\PY{p}{)}
\end{Verbatim}
\end{tcolorbox}

    \begin{Verbatim}[commandchars=\\\{\}]
This is the first line. This is the second line.
    \end{Verbatim}

    \begin{tcolorbox}[breakable, size=fbox, boxrule=1pt, pad at break*=1mm,colback=cellbackground, colframe=cellborder]
\prompt{In}{incolor}{74}{\boxspacing}
\begin{Verbatim}[commandchars=\\\{\}]
\PY{c+c1}{\PYZsh{} raw strings (r / R)}
\PY{n+nb}{print}\PY{p}{(}\PY{l+s+sa}{r}\PY{l+s+s2}{\PYZdq{}}\PY{l+s+s2}{Newlines are indicated by }\PY{l+s+s2}{\PYZbs{}}\PY{l+s+s2}{n}\PY{l+s+s2}{\PYZdq{}}\PY{p}{)}
\end{Verbatim}
\end{tcolorbox}

    \begin{Verbatim}[commandchars=\\\{\}]
Newlines are indicated by \textbackslash{}n
    \end{Verbatim}

    \begin{enumerate}
\def\labelenumi{\arabic{enumi}.}
\setcounter{enumi}{1}
\tightlist
\item
  Basic functions
\end{enumerate}

    \begin{enumerate}
\def\labelenumi{(\arabic{enumi})}
\tightlist
\item
  format()
\end{enumerate}

    \begin{itemize}
\item
  The format() method formats the specified value(s) and insert them
  inside the string's placeholder.
\item
  The placeholder is defined using curly brackets: \{\}.
\end{itemize}

    \begin{tcolorbox}[breakable, size=fbox, boxrule=1pt, pad at break*=1mm,colback=cellbackground, colframe=cellborder]
\prompt{In}{incolor}{33}{\boxspacing}
\begin{Verbatim}[commandchars=\\\{\}]
\PY{n}{age} \PY{o}{=} \PY{l+m+mi}{20}
\PY{n}{name} \PY{o}{=} \PY{l+s+s1}{\PYZsq{}}\PY{l+s+s1}{Swaroop}\PY{l+s+s1}{\PYZsq{}}
\end{Verbatim}
\end{tcolorbox}

    \textbf{Simpliy Insert value}

\begin{itemize}
\tightlist
\item
  \textbf{Method 1}: No index

  \begin{itemize}
  \tightlist
  \item
    Use ``empty \{ \}s'' as placeholders
  \item
    Insert the values in the ``( )'' of ``\{ \}.format( )'' into the
    ``empty \{ \}s'' in order
  \end{itemize}
\end{itemize}

    \begin{tcolorbox}[breakable, size=fbox, boxrule=1pt, pad at break*=1mm,colback=cellbackground, colframe=cellborder]
\prompt{In}{incolor}{59}{\boxspacing}
\begin{Verbatim}[commandchars=\\\{\}]
\PY{n+nb}{print}\PY{p}{(}\PY{l+s+s1}{\PYZsq{}}\PY{l+s+s1}{Why is }\PY{l+s+si}{\PYZob{}\PYZcb{}}\PY{l+s+s1}{ playing with that python?}\PY{l+s+se}{\PYZbs{}n}\PY{l+s+s1}{\PYZsq{}}\PY{o}{.}\PY{n}{format}\PY{p}{(}\PY{n}{name}\PY{p}{)}\PY{p}{)}

\PY{n+nb}{print}\PY{p}{(}\PY{l+s+s1}{\PYZsq{}}\PY{l+s+si}{\PYZob{}\PYZcb{}}\PY{l+s+s1}{ was }\PY{l+s+si}{\PYZob{}\PYZcb{}}\PY{l+s+s1}{ years old when he wrote this book}\PY{l+s+se}{\PYZbs{}n}\PY{l+s+s1}{\PYZsq{}}\PY{o}{.}\PY{n}{format}\PY{p}{(}\PY{n}{name}\PY{p}{,} \PY{n}{age}\PY{p}{)}\PY{p}{)}

\PY{n+nb}{print}\PY{p}{(}\PY{l+s+s1}{\PYZsq{}}\PY{l+s+si}{\PYZob{}\PYZcb{}}\PY{l+s+s1}{ years old means a lot to }\PY{l+s+si}{\PYZob{}\PYZcb{}}\PY{l+s+s1}{\PYZsq{}}\PY{o}{.}\PY{n}{format}\PY{p}{(}\PY{n}{name}\PY{p}{,} \PY{n}{age}\PY{p}{)}\PY{p}{)} 
\PY{c+c1}{\PYZsh{} Since we insert values in order, the last sentence becomes meaningless.}
\end{Verbatim}
\end{tcolorbox}

    \begin{Verbatim}[commandchars=\\\{\}]
Why is Swaroop playing with that python?

Swaroop was 20 years old when he wrote this book

Swaroop years old means a lot to 20
    \end{Verbatim}

    \begin{itemize}
\tightlist
\item
  \textbf{Method 2}: Number as index

  \begin{itemize}
  \tightlist
  \item
    Use ``\{ \}s with numbers inside'' as placeholders (e.g.: \{0\},
    \{1\}), the number inside is taken as an index
  \item
    Insert the values in the ``( )'' of ``\{ \}.format( )'' into the
    ``\{ \}s with numbers inside'' based on index, the first value
    corresponds to \{0\}, the second value corresponds to \{1\}, and so
    on.
  \end{itemize}
\end{itemize}

\textbf{Note:} Start from \textbf{zero}!!!

    \begin{tcolorbox}[breakable, size=fbox, boxrule=1pt, pad at break*=1mm,colback=cellbackground, colframe=cellborder]
\prompt{In}{incolor}{47}{\boxspacing}
\begin{Verbatim}[commandchars=\\\{\}]
\PY{n+nb}{print}\PY{p}{(}\PY{l+s+s1}{\PYZsq{}}\PY{l+s+s1}{Why is }\PY{l+s+si}{\PYZob{}0\PYZcb{}}\PY{l+s+s1}{ playing with that python?}\PY{l+s+se}{\PYZbs{}n}\PY{l+s+s1}{\PYZsq{}}\PY{o}{.}\PY{n}{format}\PY{p}{(}\PY{n}{name}\PY{p}{)}\PY{p}{)} 

\PY{n+nb}{print}\PY{p}{(}\PY{l+s+s1}{\PYZsq{}}\PY{l+s+si}{\PYZob{}0\PYZcb{}}\PY{l+s+s1}{ was }\PY{l+s+si}{\PYZob{}1\PYZcb{}}\PY{l+s+s1}{ years old when he wrote this book}\PY{l+s+se}{\PYZbs{}n}\PY{l+s+s1}{\PYZsq{}}\PY{o}{.}\PY{n}{format}\PY{p}{(}\PY{n}{name}\PY{p}{,} \PY{n}{age}\PY{p}{)}\PY{p}{)} 

\PY{n+nb}{print}\PY{p}{(}\PY{l+s+s1}{\PYZsq{}}\PY{l+s+si}{\PYZob{}1\PYZcb{}}\PY{l+s+s1}{ years old means a lot to }\PY{l+s+si}{\PYZob{}0\PYZcb{}}\PY{l+s+s1}{\PYZsq{}}\PY{o}{.}\PY{n}{format}\PY{p}{(}\PY{n}{name}\PY{p}{,} \PY{n}{age}\PY{p}{)}\PY{p}{)} 
\PY{c+c1}{\PYZsh{} Since we insert values by using number index, the last sentence becomes meaningful.}
\end{Verbatim}
\end{tcolorbox}

    \begin{Verbatim}[commandchars=\\\{\}]
Why is Swaroop playing with that python?

Swaroop was 20 years old when he wrote this book

20 years old means a lot to Swaroop
    \end{Verbatim}

    \begin{itemize}
\tightlist
\item
  \textbf{Method 3}: String as index

  \begin{itemize}
  \tightlist
  \item
    Use ``\{ \}s with words inside'' as placeholders (e.g.: \{name\},
    \{book\}), the word inside is taken as an index
  \item
    Insert the values in the ``( )'' of ``\{ \}.format( )'' into the
    ``\{ \}s with words inside'' based on index, the just like the
    dictionary.
  \end{itemize}
\end{itemize}

    \begin{tcolorbox}[breakable, size=fbox, boxrule=1pt, pad at break*=1mm,colback=cellbackground, colframe=cellborder]
\prompt{In}{incolor}{56}{\boxspacing}
\begin{Verbatim}[commandchars=\\\{\}]
\PY{n+nb}{print}\PY{p}{(}\PY{l+s+s1}{\PYZsq{}}\PY{l+s+si}{\PYZob{}name\PYZcb{}}\PY{l+s+s1}{ wrote }\PY{l+s+si}{\PYZob{}book\PYZcb{}}\PY{l+s+s1}{\PYZsq{}}\PY{o}{.}\PY{n}{format}\PY{p}{(}\PY{n}{name} \PY{o}{=} \PY{l+s+s1}{\PYZsq{}}\PY{l+s+s1}{Swaroop}\PY{l+s+s1}{\PYZsq{}}\PY{p}{,} \PY{n}{book} \PY{o}{=} \PY{l+s+s1}{\PYZsq{}}\PY{l+s+s1}{A Byte of Python}\PY{l+s+s1}{\PYZsq{}}\PY{p}{)}\PY{p}{)}
\end{Verbatim}
\end{tcolorbox}

    \begin{Verbatim}[commandchars=\\\{\}]
Swaroop wrote A Byte of Python
    \end{Verbatim}

    \begin{itemize}
\item
  \textbf{Method 4}: Dictionary as index (Literally it's still the same
  as method 3)

  It's the improved style of method 3, we just extract the dictionary
  inside the ``( )'' of ``\{ \}.format( )'' first before using the
  format() function.

  \begin{itemize}
  \tightlist
  \item
    Use ``\{ \}s with words inside'' as placeholders (e.g.: \{name\},
    \{book\}), the word inside is taken as an index
  \item
    Insert the values in the ``( )'' of ``\{ \}.format( )'' into the "\{
    \}s with the dictionary.
  \end{itemize}
\end{itemize}

    \begin{tcolorbox}[breakable, size=fbox, boxrule=1pt, pad at break*=1mm,colback=cellbackground, colframe=cellborder]
\prompt{In}{incolor}{55}{\boxspacing}
\begin{Verbatim}[commandchars=\\\{\}]
\PY{n}{site} \PY{o}{=} \PY{p}{\PYZob{}}\PY{l+s+s2}{\PYZdq{}}\PY{l+s+s2}{name}\PY{l+s+s2}{\PYZdq{}}\PY{p}{:} \PY{l+s+s2}{\PYZdq{}}\PY{l+s+s2}{programming tutorial}\PY{l+s+s2}{\PYZdq{}}\PY{p}{,} \PY{l+s+s2}{\PYZdq{}}\PY{l+s+s2}{url}\PY{l+s+s2}{\PYZdq{}}\PY{p}{:} \PY{l+s+s2}{\PYZdq{}}\PY{l+s+s2}{www.runoob.com}\PY{l+s+s2}{\PYZdq{}}\PY{p}{\PYZcb{}}

\PY{n+nb}{print}\PY{p}{(}\PY{l+s+s2}{\PYZdq{}}\PY{l+s+s2}{Website: }\PY{l+s+si}{\PYZob{}name\PYZcb{}}\PY{l+s+s2}{, Address: }\PY{l+s+si}{\PYZob{}url\PYZcb{}}\PY{l+s+se}{\PYZbs{}n}\PY{l+s+s2}{\PYZdq{}}\PY{o}{.}\PY{n}{format}\PY{p}{(}\PY{o}{*}\PY{o}{*}\PY{n}{site}\PY{p}{)}\PY{p}{)} 
\PY{c+c1}{\PYZsh{} Here, one * means it\PYZsq{}s a tuple, two ** means it\PYZsq{}s a dictionary}
\end{Verbatim}
\end{tcolorbox}

    \begin{Verbatim}[commandchars=\\\{\}]
Website: programming tutorial, Address: www.runoob.com

    \end{Verbatim}

    \begin{itemize}
\tightlist
\item
  \textbf{Method 5}: List as index (Literally it's still the same as
  method 3 \& 4)
\end{itemize}

    \begin{tcolorbox}[breakable, size=fbox, boxrule=1pt, pad at break*=1mm,colback=cellbackground, colframe=cellborder]
\prompt{In}{incolor}{58}{\boxspacing}
\begin{Verbatim}[commandchars=\\\{\}]
\PY{c+c1}{\PYZsh{}   v) 通过列表索引设置参数}
\PY{n}{my\PYZus{}list} \PY{o}{=} \PY{p}{[}\PY{l+s+s1}{\PYZsq{}}\PY{l+s+s1}{programming tutorial}\PY{l+s+s1}{\PYZsq{}}\PY{p}{,} \PY{l+s+s1}{\PYZsq{}}\PY{l+s+s1}{www.runoob.com}\PY{l+s+s1}{\PYZsq{}}\PY{p}{]}
\PY{n+nb}{print}\PY{p}{(}\PY{l+s+s2}{\PYZdq{}}\PY{l+s+s2}{Website: }\PY{l+s+si}{\PYZob{}0[0]\PYZcb{}}\PY{l+s+s2}{, Address: }\PY{l+s+si}{\PYZob{}0[1]\PYZcb{}}\PY{l+s+se}{\PYZbs{}n}\PY{l+s+s2}{\PYZdq{}}\PY{o}{.}\PY{n}{format}\PY{p}{(}\PY{n}{my\PYZus{}list}\PY{p}{)}\PY{p}{)}  \PY{c+c1}{\PYZsh{} \PYZdq{}0\PYZdq{} is a must here}
\end{Verbatim}
\end{tcolorbox}

    \begin{Verbatim}[commandchars=\\\{\}]
Website: programming tutorial, Address: www.runoob.com

    \end{Verbatim}

    \begin{itemize}
\tightlist
\item
  We can also use ``+'' to combine the strings without using format( ),
  but its drawback is obvious when we have lots of strings needed to be
  combined.
\end{itemize}

    \begin{tcolorbox}[breakable, size=fbox, boxrule=1pt, pad at break*=1mm,colback=cellbackground, colframe=cellborder]
\prompt{In}{incolor}{37}{\boxspacing}
\begin{Verbatim}[commandchars=\\\{\}]
\PY{c+c1}{\PYZsh{} [Notes: use plus sign (\PYZsq{}+\PYZsq{}) can also have the same effect (but obviously more complex)]}
\PY{n+nb}{print}\PY{p}{(}\PY{n}{name} \PY{o}{+} \PY{l+s+s1}{\PYZsq{}}\PY{l+s+s1}{is}\PY{l+s+s1}{\PYZsq{}} \PY{o}{+} \PY{n+nb}{str}\PY{p}{(}\PY{n}{age}\PY{p}{)} \PY{o}{+} \PY{l+s+s1}{\PYZsq{}}\PY{l+s+s1}{years old}\PY{l+s+s1}{\PYZsq{}}\PY{p}{)}
\end{Verbatim}
\end{tcolorbox}

    \begin{Verbatim}[commandchars=\\\{\}]
Swaroopis20years old
    \end{Verbatim}

    \textbf{Insert value - More difficult situation}

\begin{itemize}
\tightlist
\item
  \textbf{Situation 1}: Insert Number with fixed type
\end{itemize}

    \begin{tcolorbox}[breakable, size=fbox, boxrule=1pt, pad at break*=1mm,colback=cellbackground, colframe=cellborder]
\prompt{In}{incolor}{63}{\boxspacing}
\begin{Verbatim}[commandchars=\\\{\}]
\PY{c+c1}{\PYZsh{} the number is rounded to three decimal places}
\PY{n+nb}{print}\PY{p}{(}\PY{l+s+s1}{\PYZsq{}}\PY{l+s+si}{\PYZob{}0:.3f\PYZcb{}}\PY{l+s+s1}{\PYZsq{}}\PY{o}{.}\PY{n}{format}\PY{p}{(}\PY{l+m+mf}{1.0}\PY{o}{/}\PY{l+m+mi}{3}\PY{p}{)}\PY{p}{)} 

\PY{c+c1}{\PYZsh{} the number is rounded to three decimal places \PYZam{} keep the \PYZdq{}negative or positive\PYZdq{} sign}
\PY{n+nb}{print}\PY{p}{(}\PY{l+s+s1}{\PYZsq{}}\PY{l+s+si}{\PYZob{}0:+.3f\PYZcb{}}\PY{l+s+s1}{\PYZsq{}}\PY{o}{.}\PY{n}{format}\PY{p}{(}\PY{l+m+mf}{1.0}\PY{o}{/}\PY{l+m+mi}{3}\PY{p}{)}\PY{p}{)} 

\PY{n+nb}{print}\PY{p}{(}\PY{l+s+s1}{\PYZsq{}}\PY{l+s+si}{\PYZob{}0:+.3f\PYZcb{}}\PY{l+s+s1}{\PYZsq{}}\PY{o}{.}\PY{n}{format}\PY{p}{(}\PY{o}{\PYZhy{}}\PY{l+m+mf}{1.0}\PY{o}{/}\PY{l+m+mi}{3}\PY{p}{)}\PY{p}{)} 
\end{Verbatim}
\end{tcolorbox}

    \begin{Verbatim}[commandchars=\\\{\}]
0.333
+0.333
-0.333
    \end{Verbatim}

    \begin{itemize}
\tightlist
\item
  \textbf{Situation 2}: Insert some special signs with fixed type

  \begin{itemize}
  \tightlist
  \item
    insert ``the special sign'' before ``\^{}'' or ``\textgreater{}'' or
    ``\textless{}'' to the string which is in the ``( ) of format( )''

    \begin{itemize}
    \tightlist
    \item
      ``\^{}'': insert from both sides of the string, both left side and
      right side
    \item
      ``\textgreater{}'': insert only from right side
    \item
      ``\textless{}'': insert only from left side
    \end{itemize}
  \end{itemize}
\end{itemize}

    \begin{tcolorbox}[breakable, size=fbox, boxrule=1pt, pad at break*=1mm,colback=cellbackground, colframe=cellborder]
\prompt{In}{incolor}{69}{\boxspacing}
\begin{Verbatim}[commandchars=\\\{\}]
\PY{c+c1}{\PYZsh{} \PYZdq{}\PYZca{}\PYZdq{}: insert from both sides}
\PY{n+nb}{print}\PY{p}{(}\PY{l+s+s1}{\PYZsq{}}\PY{l+s+si}{\PYZob{}0:\PYZus{}\PYZca{}11\PYZcb{}}\PY{l+s+s1}{\PYZsq{}}\PY{o}{.}\PY{n}{format}\PY{p}{(}\PY{l+s+s1}{\PYZsq{}}\PY{l+s+s1}{hello}\PY{l+s+s1}{\PYZsq{}}\PY{p}{)}\PY{p}{)}

\PY{c+c1}{\PYZsh{} \PYZdq{}\PYZgt{}\PYZdq{}: insert only from right side}
\PY{n+nb}{print}\PY{p}{(}\PY{l+s+s1}{\PYZsq{}}\PY{l+s+si}{\PYZob{}0:\PYZus{}\PYZlt{}11\PYZcb{}}\PY{l+s+s1}{\PYZsq{}}\PY{o}{.}\PY{n}{format}\PY{p}{(}\PY{l+s+s1}{\PYZsq{}}\PY{l+s+s1}{hello}\PY{l+s+s1}{\PYZsq{}}\PY{p}{)}\PY{p}{)}

\PY{c+c1}{\PYZsh{} \PYZdq{}\PYZlt{}\PYZdq{}: insert only from left side}
\PY{n+nb}{print}\PY{p}{(}\PY{l+s+s1}{\PYZsq{}}\PY{l+s+si}{\PYZob{}0:\PYZus{}\PYZgt{}11\PYZcb{}}\PY{l+s+s1}{\PYZsq{}}\PY{o}{.}\PY{n}{format}\PY{p}{(}\PY{l+s+s1}{\PYZsq{}}\PY{l+s+s1}{hello}\PY{l+s+s1}{\PYZsq{}}\PY{p}{)}\PY{p}{)}
\end{Verbatim}
\end{tcolorbox}

    \begin{Verbatim}[commandchars=\\\{\}]
\_\_\_hello\_\_\_
hello\_\_\_\_\_\_
\_\_\_\_\_\_hello
    \end{Verbatim}

    \begin{itemize}
\tightlist
\item
  \textbf{Situation 2}: Insert some special signs with fixed type

  \begin{itemize}
  \tightlist
  \item
    the alpha after ``:'' represents different types of number system,
    base 10 (decimal), base 2 (binary)
  \end{itemize}
\end{itemize}

    \begin{tcolorbox}[breakable, size=fbox, boxrule=1pt, pad at break*=1mm,colback=cellbackground, colframe=cellborder]
\prompt{In}{incolor}{70}{\boxspacing}
\begin{Verbatim}[commandchars=\\\{\}]
\PY{c+c1}{\PYZsh{}   iii) 表示进制 }
\PY{n+nb}{print}\PY{p}{(}\PY{l+s+s1}{\PYZsq{}}\PY{l+s+si}{\PYZob{}:b\PYZcb{}}\PY{l+s+s1}{\PYZsq{}}\PY{o}{.}\PY{n}{format}\PY{p}{(}\PY{l+m+mi}{11}\PY{p}{)}\PY{p}{)}
\PY{n+nb}{print}\PY{p}{(}\PY{l+s+s1}{\PYZsq{}}\PY{l+s+si}{\PYZob{}:d\PYZcb{}}\PY{l+s+s1}{\PYZsq{}}\PY{o}{.}\PY{n}{format}\PY{p}{(}\PY{l+m+mi}{11}\PY{p}{)}\PY{p}{)}
\PY{n+nb}{print}\PY{p}{(}\PY{l+s+s1}{\PYZsq{}}\PY{l+s+si}{\PYZob{}:o\PYZcb{}}\PY{l+s+s1}{\PYZsq{}}\PY{o}{.}\PY{n}{format}\PY{p}{(}\PY{l+m+mi}{11}\PY{p}{)}\PY{p}{)}
\PY{n+nb}{print}\PY{p}{(}\PY{l+s+s1}{\PYZsq{}}\PY{l+s+si}{\PYZob{}:x\PYZcb{}}\PY{l+s+s1}{\PYZsq{}}\PY{o}{.}\PY{n}{format}\PY{p}{(}\PY{l+m+mi}{11}\PY{p}{)}\PY{p}{)}
\PY{n+nb}{print}\PY{p}{(}\PY{l+s+s1}{\PYZsq{}}\PY{l+s+si}{\PYZob{}:\PYZsh{}x\PYZcb{}}\PY{l+s+s1}{\PYZsq{}}\PY{o}{.}\PY{n}{format}\PY{p}{(}\PY{l+m+mi}{11}\PY{p}{)}\PY{p}{)}
\PY{n+nb}{print}\PY{p}{(}\PY{l+s+s1}{\PYZsq{}}\PY{l+s+si}{\PYZob{}:\PYZsh{}X\PYZcb{}}\PY{l+s+s1}{\PYZsq{}}\PY{o}{.}\PY{n}{format}\PY{p}{(}\PY{l+m+mi}{11}\PY{p}{)}\PY{p}{)}
\end{Verbatim}
\end{tcolorbox}

    \begin{Verbatim}[commandchars=\\\{\}]
1011
11
13
b
0xb
0XB
    \end{Verbatim}

    \begin{enumerate}
\def\labelenumi{(\arabic{enumi})}
\setcounter{enumi}{1}
\tightlist
\item
  print()
\end{enumerate}

    \begin{tcolorbox}[breakable, size=fbox, boxrule=1pt, pad at break*=1mm,colback=cellbackground, colframe=cellborder]
\prompt{In}{incolor}{66}{\boxspacing}
\begin{Verbatim}[commandchars=\\\{\}]
\PY{c+c1}{\PYZsh{} without a space}
\PY{n+nb}{print}\PY{p}{(}\PY{l+s+s1}{\PYZsq{}}\PY{l+s+s1}{a}\PY{l+s+s1}{\PYZsq{}}\PY{p}{,} \PY{n}{end} \PY{o}{=} \PY{l+s+s1}{\PYZsq{}}\PY{l+s+s1}{\PYZsq{}}\PY{p}{)}
\PY{n+nb}{print}\PY{p}{(}\PY{l+s+s1}{\PYZsq{}}\PY{l+s+s1}{b}\PY{l+s+s1}{\PYZsq{}}\PY{p}{,} \PY{n}{end} \PY{o}{=} \PY{l+s+s1}{\PYZsq{}}\PY{l+s+s1}{\PYZsq{}}\PY{p}{)}
\PY{c+c1}{\PYZsh{} with a space}
\PY{n+nb}{print}\PY{p}{(}\PY{l+s+s1}{\PYZsq{}}\PY{l+s+s1}{a}\PY{l+s+s1}{\PYZsq{}}\PY{p}{,} \PY{n}{end} \PY{o}{=} \PY{l+s+s1}{\PYZsq{}}\PY{l+s+s1}{ }\PY{l+s+s1}{\PYZsq{}}\PY{p}{)}
\PY{n+nb}{print}\PY{p}{(}\PY{l+s+s1}{\PYZsq{}}\PY{l+s+s1}{b}\PY{l+s+s1}{\PYZsq{}}\PY{p}{,} \PY{n}{end} \PY{o}{=} \PY{l+s+s1}{\PYZsq{}}\PY{l+s+s1}{ }\PY{l+s+s1}{\PYZsq{}}\PY{p}{)}
\PY{n+nb}{print}\PY{p}{(}\PY{l+s+s1}{\PYZsq{}}\PY{l+s+s1}{c}\PY{l+s+s1}{\PYZsq{}}\PY{p}{)}
\end{Verbatim}
\end{tcolorbox}

    \begin{Verbatim}[commandchars=\\\{\}]
aba b c
    \end{Verbatim}

    \begin{enumerate}
\def\labelenumi{(\arabic{enumi})}
\setcounter{enumi}{2}
\tightlist
\item
  range()
\end{enumerate}

    \begin{itemize}
\tightlist
\item
  range(a,b): generate {[}a, a+1, a+2, \ldots, b-1{]}
\item
  range(a,b,c): generate {[}a, a+c, a+2c, \ldots, a+nc(\textless=
  b-1){]}
\end{itemize}

But range shows one number on one time, so if you want to show all the
numbers shown in this range, you have to use list

    \begin{tcolorbox}[breakable, size=fbox, boxrule=1pt, pad at break*=1mm,colback=cellbackground, colframe=cellborder]
\prompt{In}{incolor}{67}{\boxspacing}
\begin{Verbatim}[commandchars=\\\{\}]
\PY{n+nb}{print}\PY{p}{(}\PY{n+nb}{list}\PY{p}{(}\PY{n+nb}{range}\PY{p}{(}\PY{l+m+mi}{1}\PY{p}{,}\PY{l+m+mi}{5}\PY{p}{)}\PY{p}{)}\PY{p}{)}
\PY{n+nb}{print}\PY{p}{(}\PY{n+nb}{range}\PY{p}{(}\PY{l+m+mi}{1}\PY{p}{,}\PY{l+m+mi}{5}\PY{p}{)}\PY{p}{)}
\end{Verbatim}
\end{tcolorbox}

    \begin{Verbatim}[commandchars=\\\{\}]
[1, 2, 3, 4]
range(1, 5)
    \end{Verbatim}

    \begin{enumerate}
\def\labelenumi{(\arabic{enumi})}
\setcounter{enumi}{3}
\tightlist
\item
  id()
\end{enumerate}

    \begin{itemize}
\tightlist
\item
  what the usage of this function?
\item
  id() is to get the storage direction of this character
\end{itemize}

    \begin{tcolorbox}[breakable, size=fbox, boxrule=1pt, pad at break*=1mm,colback=cellbackground, colframe=cellborder]
\prompt{In}{incolor}{68}{\boxspacing}
\begin{Verbatim}[commandchars=\\\{\}]
\PY{n}{a} \PY{o}{=} \PY{l+s+s1}{\PYZsq{}}\PY{l+s+s1}{runoob}\PY{l+s+s1}{\PYZsq{}}
\PY{n+nb}{print}\PY{p}{(}\PY{n+nb}{id}\PY{p}{(}\PY{n}{a}\PY{p}{)}\PY{p}{)}
\PY{n}{b} \PY{o}{=} \PY{l+m+mi}{1}
\PY{n+nb}{print}\PY{p}{(}\PY{n+nb}{id}\PY{p}{(}\PY{n}{b}\PY{p}{)}\PY{p}{)}
\end{Verbatim}
\end{tcolorbox}

    \begin{Verbatim}[commandchars=\\\{\}]
140264836560368
4334315872
    \end{Verbatim}

    \begin{enumerate}
\def\labelenumi{(\arabic{enumi})}
\setcounter{enumi}{4}
\tightlist
\item
  functions in random package
\end{enumerate}

    \begin{tcolorbox}[breakable, size=fbox, boxrule=1pt, pad at break*=1mm,colback=cellbackground, colframe=cellborder]
\prompt{In}{incolor}{43}{\boxspacing}
\begin{Verbatim}[commandchars=\\\{\}]
\PY{k+kn}{import} \PY{n+nn}{random}

\PY{n+nb}{print}\PY{p}{(}\PY{n}{random}\PY{o}{.}\PY{n}{randint}\PY{p}{(}\PY{l+m+mi}{1}\PY{p}{,}\PY{l+m+mi}{10}\PY{p}{)}\PY{p}{)}        \PY{c+c1}{\PYZsh{} generate a random integer between 1 and 10 }
\PY{n+nb}{print}\PY{p}{(}\PY{n}{random}\PY{o}{.}\PY{n}{random}\PY{p}{(}\PY{p}{)}\PY{p}{)}             \PY{c+c1}{\PYZsh{} generate a random float between 0 and 1}
\PY{n+nb}{print}\PY{p}{(}\PY{n}{random}\PY{o}{.}\PY{n}{uniform}\PY{p}{(}\PY{l+m+mf}{1.1}\PY{p}{,}\PY{l+m+mf}{5.4}\PY{p}{)}\PY{p}{)}     \PY{c+c1}{\PYZsh{} generate a random float between 1.1 and 5.4}
\PY{n+nb}{print}\PY{p}{(}\PY{n}{random}\PY{o}{.}\PY{n}{choice}\PY{p}{(}\PY{l+s+s1}{\PYZsq{}}\PY{l+s+s1}{tomorrow}\PY{l+s+s1}{\PYZsq{}}\PY{p}{)}\PY{p}{)}   \PY{c+c1}{\PYZsh{} pick a random element from the sequence }
\PY{n+nb}{print}\PY{p}{(}\PY{n}{random}\PY{o}{.}\PY{n}{randrange}\PY{p}{(}\PY{l+m+mi}{1}\PY{p}{,}\PY{l+m+mi}{100}\PY{p}{,}\PY{l+m+mi}{2}\PY{p}{)}\PY{p}{)}   \PY{c+c1}{\PYZsh{} generate a random integer from the list beginning from 1 to 100 with the step as 2 }
\end{Verbatim}
\end{tcolorbox}

    \begin{Verbatim}[commandchars=\\\{\}]
1
0.16821664991489804
2.026160535358066
r
49
    \end{Verbatim}

    \begin{tcolorbox}[breakable, size=fbox, boxrule=1pt, pad at break*=1mm,colback=cellbackground, colframe=cellborder]
\prompt{In}{incolor}{44}{\boxspacing}
\begin{Verbatim}[commandchars=\\\{\}]
\PY{n}{a}\PY{o}{=}\PY{p}{[}\PY{l+m+mi}{1}\PY{p}{,}\PY{l+m+mi}{3}\PY{p}{,}\PY{l+m+mi}{5}\PY{p}{,}\PY{l+m+mi}{6}\PY{p}{,}\PY{l+m+mi}{7}\PY{p}{]}                \PY{c+c1}{\PYZsh{} shuffle the order in list a }
\PY{n}{random}\PY{o}{.}\PY{n}{shuffle}\PY{p}{(}\PY{n}{a}\PY{p}{)}
\PY{n+nb}{print}\PY{p}{(}\PY{n}{a}\PY{p}{)}
\end{Verbatim}
\end{tcolorbox}

    \begin{Verbatim}[commandchars=\\\{\}]
[6, 5, 3, 7, 1]
    \end{Verbatim}

    \begin{enumerate}
\def\labelenumi{\arabic{enumi}.}
\setcounter{enumi}{2}
\tightlist
\item
  Basic regulations about naming variables
\end{enumerate}

    In Python, we can use English character, numbers and \_ to build our
identifier, or in other words, to name variables. However, we can
\textbf{not} use numbers to start a identifier.

Besides, the identifiers starting with \_ also have special meanings, we
can not simpliy to use ``from \ldots{} import \ldots.'' to import
packages.

What's more, if this identifiers starting with two \_s, like \_\_foo or
\_\_foo\_\_, it's another special method to identify the vairables. We
need to be careful with that.

?? what if -y = 50 / 2y = 50 (2y is the variable name or means 2
multipled by y?)

?? what if y = 2x - f (2x is the variable name or means 2 multipled by
x?)

    Part 1. Python as a calculator

    \begin{enumerate}
\def\labelenumi{\arabic{enumi}.}
\tightlist
\item
  Basic types of calculation
\end{enumerate}

    \begin{enumerate}
\def\labelenumi{(\arabic{enumi})}
\tightlist
\item
  Addition: +
\end{enumerate}

    \begin{tcolorbox}[breakable, size=fbox, boxrule=1pt, pad at break*=1mm,colback=cellbackground, colframe=cellborder]
\prompt{In}{incolor}{84}{\boxspacing}
\begin{Verbatim}[commandchars=\\\{\}]
\PY{n+nb}{print}\PY{p}{(}\PY{l+m+mi}{3} \PY{o}{+} \PY{l+m+mi}{5}\PY{p}{)}
\PY{n+nb}{print}\PY{p}{(}\PY{l+s+s1}{\PYZsq{}}\PY{l+s+s1}{a}\PY{l+s+s1}{\PYZsq{}} \PY{o}{+} \PY{l+s+s1}{\PYZsq{}}\PY{l+s+s1}{b}\PY{l+s+s1}{\PYZsq{}}\PY{p}{)} \PY{c+c1}{\PYZsh{} (link two strings)}
\end{Verbatim}
\end{tcolorbox}

    \begin{Verbatim}[commandchars=\\\{\}]
8
ab
    \end{Verbatim}

    \begin{enumerate}
\def\labelenumi{(\arabic{enumi})}
\setcounter{enumi}{1}
\tightlist
\item
  Subtraction: -
\end{enumerate}

    \begin{tcolorbox}[breakable, size=fbox, boxrule=1pt, pad at break*=1mm,colback=cellbackground, colframe=cellborder]
\prompt{In}{incolor}{85}{\boxspacing}
\begin{Verbatim}[commandchars=\\\{\}]
\PY{n+nb}{print}\PY{p}{(}\PY{l+m+mi}{10} \PY{o}{\PYZhy{}} \PY{l+m+mi}{7}\PY{p}{)}
\PY{n+nb}{print}\PY{p}{(} \PY{o}{\PYZhy{}} \PY{l+m+mf}{5.2}\PY{p}{)} \PY{c+c1}{\PYZsh{} (if the first number is not shown, then set it as zero)}
\end{Verbatim}
\end{tcolorbox}

    \begin{Verbatim}[commandchars=\\\{\}]
3
-5.2
    \end{Verbatim}

    \begin{enumerate}
\def\labelenumi{(\arabic{enumi})}
\setcounter{enumi}{2}
\tightlist
\item
  Multiplication: *
\end{enumerate}

    \begin{tcolorbox}[breakable, size=fbox, boxrule=1pt, pad at break*=1mm,colback=cellbackground, colframe=cellborder]
\prompt{In}{incolor}{86}{\boxspacing}
\begin{Verbatim}[commandchars=\\\{\}]
\PY{n+nb}{print}\PY{p}{(}\PY{l+m+mi}{2} \PY{o}{*} \PY{l+m+mi}{3}\PY{p}{)}
\PY{n+nb}{print}\PY{p}{(}\PY{l+s+s1}{\PYZsq{}}\PY{l+s+s1}{la}\PY{l+s+s1}{\PYZsq{}} \PY{o}{*} \PY{l+m+mi}{3}\PY{p}{)} \PY{c+c1}{\PYZsh{} (repeat the first string)}
\end{Verbatim}
\end{tcolorbox}

    \begin{Verbatim}[commandchars=\\\{\}]
6
lalala
    \end{Verbatim}

    \begin{enumerate}
\def\labelenumi{(\arabic{enumi})}
\setcounter{enumi}{3}
\tightlist
\item
  Division: /
\end{enumerate}

    \begin{tcolorbox}[breakable, size=fbox, boxrule=1pt, pad at break*=1mm,colback=cellbackground, colframe=cellborder]
\prompt{In}{incolor}{87}{\boxspacing}
\begin{Verbatim}[commandchars=\\\{\}]
\PY{n+nb}{print}\PY{p}{(}\PY{l+m+mi}{5} \PY{o}{/} \PY{l+m+mi}{8}\PY{p}{)}
\PY{n+nb}{print}\PY{p}{(}\PY{l+m+mi}{13} \PY{o}{/} \PY{l+m+mi}{3}\PY{p}{)}
\end{Verbatim}
\end{tcolorbox}

    \begin{Verbatim}[commandchars=\\\{\}]
0.625
4.333333333333333
    \end{Verbatim}

    \begin{enumerate}
\def\labelenumi{(\arabic{enumi})}
\setcounter{enumi}{4}
\tightlist
\item
  Exponentiation: **
\end{enumerate}

    \begin{tcolorbox}[breakable, size=fbox, boxrule=1pt, pad at break*=1mm,colback=cellbackground, colframe=cellborder]
\prompt{In}{incolor}{88}{\boxspacing}
\begin{Verbatim}[commandchars=\\\{\}]
\PY{n+nb}{print}\PY{p}{(}\PY{l+m+mi}{2} \PY{o}{*}\PY{o}{*} \PY{l+m+mi}{3}\PY{p}{)}
\PY{n+nb}{print}\PY{p}{(}\PY{l+m+mi}{3} \PY{o}{*}\PY{o}{*} \PY{l+m+mi}{4}\PY{p}{)} \PY{c+c1}{\PYZsh{} (Remember: NOT \PYZsq{}\PYZca{}\PYZsq{} !!!)}
\end{Verbatim}
\end{tcolorbox}

    \begin{Verbatim}[commandchars=\\\{\}]
8
81
    \end{Verbatim}

    \begin{enumerate}
\def\labelenumi{(\arabic{enumi})}
\setcounter{enumi}{5}
\tightlist
\item
  Round down: //
\end{enumerate}

    \begin{tcolorbox}[breakable, size=fbox, boxrule=1pt, pad at break*=1mm,colback=cellbackground, colframe=cellborder]
\prompt{In}{incolor}{89}{\boxspacing}
\begin{Verbatim}[commandchars=\\\{\}]
\PY{n+nb}{print}\PY{p}{(}\PY{l+m+mi}{13} \PY{o}{/}\PY{o}{/} \PY{l+m+mi}{3}\PY{p}{)}
\PY{n+nb}{print}\PY{p}{(}\PY{o}{\PYZhy{}}\PY{l+m+mi}{13} \PY{o}{/}\PY{o}{/} \PY{l+m+mi}{3}\PY{p}{)}
\end{Verbatim}
\end{tcolorbox}

    \begin{Verbatim}[commandchars=\\\{\}]
4
-5
    \end{Verbatim}

    \begin{enumerate}
\def\labelenumi{(\arabic{enumi})}
\setcounter{enumi}{6}
\tightlist
\item
  Modulo: \%
\end{enumerate}

    \begin{tcolorbox}[breakable, size=fbox, boxrule=1pt, pad at break*=1mm,colback=cellbackground, colframe=cellborder]
\prompt{In}{incolor}{90}{\boxspacing}
\begin{Verbatim}[commandchars=\\\{\}]
\PY{n+nb}{print}\PY{p}{(}\PY{l+m+mi}{13} \PY{o}{\PYZpc{}} \PY{l+m+mi}{3}\PY{p}{)}
\PY{n+nb}{print}\PY{p}{(}\PY{o}{\PYZhy{}}\PY{l+m+mf}{25.5} \PY{o}{\PYZpc{}} \PY{l+m+mf}{2.25}\PY{p}{)}
\end{Verbatim}
\end{tcolorbox}

    \begin{Verbatim}[commandchars=\\\{\}]
1
1.5
    \end{Verbatim}

    \begin{enumerate}
\def\labelenumi{\arabic{enumi}.}
\setcounter{enumi}{1}
\tightlist
\item
  Special types of calculation (relevant to decimal system and Binary
  system)
\end{enumerate}

    \begin{enumerate}
\def\labelenumi{(\arabic{enumi})}
\tightlist
\item
  Left move: \textless\textless{}
\end{enumerate}

    \begin{tcolorbox}[breakable, size=fbox, boxrule=1pt, pad at break*=1mm,colback=cellbackground, colframe=cellborder]
\prompt{In}{incolor}{93}{\boxspacing}
\begin{Verbatim}[commandchars=\\\{\}]
\PY{n+nb}{print}\PY{p}{(}\PY{l+m+mi}{2} \PY{o}{\PYZlt{}\PYZlt{}} \PY{l+m+mi}{2}\PY{p}{)}
\end{Verbatim}
\end{tcolorbox}

    \begin{Verbatim}[commandchars=\\\{\}]
8
    \end{Verbatim}

    \begin{enumerate}
\def\labelenumi{(\arabic{enumi})}
\setcounter{enumi}{1}
\tightlist
\item
  Right move: \textgreater\textgreater{}
\end{enumerate}

    \begin{tcolorbox}[breakable, size=fbox, boxrule=1pt, pad at break*=1mm,colback=cellbackground, colframe=cellborder]
\prompt{In}{incolor}{94}{\boxspacing}
\begin{Verbatim}[commandchars=\\\{\}]
\PY{n+nb}{print}\PY{p}{(}\PY{l+m+mi}{11} \PY{o}{\PYZgt{}\PYZgt{}} \PY{l+m+mi}{1}\PY{p}{)}
\end{Verbatim}
\end{tcolorbox}

    \begin{Verbatim}[commandchars=\\\{\}]
5
    \end{Verbatim}

    \begin{enumerate}
\def\labelenumi{(\arabic{enumi})}
\setcounter{enumi}{2}
\tightlist
\item
  Bitwise AND: \&
\end{enumerate}

    \begin{tcolorbox}[breakable, size=fbox, boxrule=1pt, pad at break*=1mm,colback=cellbackground, colframe=cellborder]
\prompt{In}{incolor}{96}{\boxspacing}
\begin{Verbatim}[commandchars=\\\{\}]
\PY{n+nb}{print}\PY{p}{(}\PY{l+m+mi}{5} \PY{o}{\PYZam{}} \PY{l+m+mi}{3}\PY{p}{)}
\end{Verbatim}
\end{tcolorbox}

    \begin{Verbatim}[commandchars=\\\{\}]
1
    \end{Verbatim}

    \begin{enumerate}
\def\labelenumi{(\arabic{enumi})}
\setcounter{enumi}{3}
\tightlist
\item
  Bitwise OR: \textbar{}
\end{enumerate}

    \begin{tcolorbox}[breakable, size=fbox, boxrule=1pt, pad at break*=1mm,colback=cellbackground, colframe=cellborder]
\prompt{In}{incolor}{97}{\boxspacing}
\begin{Verbatim}[commandchars=\\\{\}]
\PY{n+nb}{print}\PY{p}{(}\PY{l+m+mi}{5} \PY{o}{|} \PY{l+m+mi}{3}\PY{p}{)}
\end{Verbatim}
\end{tcolorbox}

    \begin{Verbatim}[commandchars=\\\{\}]
7
    \end{Verbatim}

    \begin{enumerate}
\def\labelenumi{(\arabic{enumi})}
\setcounter{enumi}{4}
\tightlist
\item
  Bitwise XOR: \^{}
\end{enumerate}

    \begin{tcolorbox}[breakable, size=fbox, boxrule=1pt, pad at break*=1mm,colback=cellbackground, colframe=cellborder]
\prompt{In}{incolor}{98}{\boxspacing}
\begin{Verbatim}[commandchars=\\\{\}]
\PY{n+nb}{print}\PY{p}{(}\PY{l+m+mi}{5} \PY{o}{\PYZca{}} \PY{l+m+mi}{3}\PY{p}{)}
\end{Verbatim}
\end{tcolorbox}

    \begin{Verbatim}[commandchars=\\\{\}]
6
    \end{Verbatim}

    \begin{enumerate}
\def\labelenumi{(\arabic{enumi})}
\setcounter{enumi}{5}
\tightlist
\item
  compl: \textasciitilde{}
\end{enumerate}

    \begin{tcolorbox}[breakable, size=fbox, boxrule=1pt, pad at break*=1mm,colback=cellbackground, colframe=cellborder]
\prompt{In}{incolor}{99}{\boxspacing}
\begin{Verbatim}[commandchars=\\\{\}]
\PY{n+nb}{print}\PY{p}{(}\PY{o}{\PYZti{}}\PY{l+m+mi}{5}\PY{p}{)}
\end{Verbatim}
\end{tcolorbox}

    \begin{Verbatim}[commandchars=\\\{\}]
-6
    \end{Verbatim}

    \begin{enumerate}
\def\labelenumi{\arabic{enumi}.}
\setcounter{enumi}{2}
\tightlist
\item
  Special types of calculation (relevant to boolean variable)
\end{enumerate}

    \begin{enumerate}
\def\labelenumi{(\arabic{enumi})}
\tightlist
\item
  Smaller than: \textless{}
\end{enumerate}

    \begin{tcolorbox}[breakable, size=fbox, boxrule=1pt, pad at break*=1mm,colback=cellbackground, colframe=cellborder]
\prompt{In}{incolor}{100}{\boxspacing}
\begin{Verbatim}[commandchars=\\\{\}]
\PY{n+nb}{print}\PY{p}{(}\PY{l+m+mi}{5} \PY{o}{\PYZlt{}} \PY{l+m+mi}{3}\PY{p}{)}
\PY{n+nb}{print}\PY{p}{(}\PY{l+m+mi}{3} \PY{o}{\PYZlt{}} \PY{l+m+mi}{5}\PY{p}{)}
\end{Verbatim}
\end{tcolorbox}

    \begin{Verbatim}[commandchars=\\\{\}]
False
True
    \end{Verbatim}

    \begin{enumerate}
\def\labelenumi{(\arabic{enumi})}
\setcounter{enumi}{1}
\tightlist
\item
  Larger than: \textgreater{}
\end{enumerate}

    \begin{tcolorbox}[breakable, size=fbox, boxrule=1pt, pad at break*=1mm,colback=cellbackground, colframe=cellborder]
\prompt{In}{incolor}{101}{\boxspacing}
\begin{Verbatim}[commandchars=\\\{\}]
\PY{n+nb}{print}\PY{p}{(}\PY{l+m+mi}{5} \PY{o}{\PYZgt{}} \PY{l+m+mi}{3}\PY{p}{)}
\PY{n+nb}{print}\PY{p}{(}\PY{l+m+mi}{3} \PY{o}{\PYZgt{}} \PY{l+m+mi}{5}\PY{p}{)}
\end{Verbatim}
\end{tcolorbox}

    \begin{Verbatim}[commandchars=\\\{\}]
True
False
    \end{Verbatim}

    \begin{enumerate}
\def\labelenumi{(\arabic{enumi})}
\setcounter{enumi}{2}
\tightlist
\item
  Smaller than or equal to: \textless=
\end{enumerate}

    \begin{tcolorbox}[breakable, size=fbox, boxrule=1pt, pad at break*=1mm,colback=cellbackground, colframe=cellborder]
\prompt{In}{incolor}{102}{\boxspacing}
\begin{Verbatim}[commandchars=\\\{\}]
\PY{n}{x} \PY{o}{=} \PY{l+m+mi}{31}\PY{p}{;} \PY{n}{y} \PY{o}{=} \PY{l+m+mi}{6}\PY{p}{;} \PY{n+nb}{print}\PY{p}{(}\PY{n}{x} \PY{o}{\PYZlt{}}\PY{o}{=} \PY{n}{y}\PY{p}{)}
\end{Verbatim}
\end{tcolorbox}

    \begin{Verbatim}[commandchars=\\\{\}]
False
    \end{Verbatim}

    \begin{enumerate}
\def\labelenumi{(\arabic{enumi})}
\setcounter{enumi}{3}
\tightlist
\item
  Larger than or equal to: \textgreater=
\end{enumerate}

    \begin{tcolorbox}[breakable, size=fbox, boxrule=1pt, pad at break*=1mm,colback=cellbackground, colframe=cellborder]
\prompt{In}{incolor}{103}{\boxspacing}
\begin{Verbatim}[commandchars=\\\{\}]
\PY{n}{a} \PY{o}{=} \PY{l+m+mi}{4}\PY{p}{;} \PY{n}{b} \PY{o}{=} \PY{l+m+mi}{3}\PY{p}{;} \PY{n+nb}{print}\PY{p}{(}\PY{n}{a} \PY{o}{\PYZgt{}}\PY{o}{=} \PY{n}{b}\PY{p}{)}
\end{Verbatim}
\end{tcolorbox}

    \begin{Verbatim}[commandchars=\\\{\}]
True
    \end{Verbatim}

    \begin{enumerate}
\def\labelenumi{(\arabic{enumi})}
\setcounter{enumi}{4}
\tightlist
\item
  Equal to: ==
\end{enumerate}

    \begin{tcolorbox}[breakable, size=fbox, boxrule=1pt, pad at break*=1mm,colback=cellbackground, colframe=cellborder]
\prompt{In}{incolor}{104}{\boxspacing}
\begin{Verbatim}[commandchars=\\\{\}]
\PY{c+c1}{\PYZsh{} (!!! two equation signs!!!)}
\PY{n}{m} \PY{o}{=} \PY{l+m+mi}{2}\PY{p}{;} \PY{n}{n} \PY{o}{=} \PY{l+m+mi}{2}\PY{p}{;} \PY{n+nb}{print}\PY{p}{(}\PY{n}{m} \PY{o}{==} \PY{n}{n}\PY{p}{)}
\PY{n}{r} \PY{o}{=} \PY{l+s+s1}{\PYZsq{}}\PY{l+s+s1}{str}\PY{l+s+s1}{\PYZsq{}}\PY{p}{;} \PY{n}{t} \PY{o}{=} \PY{l+s+s1}{\PYZsq{}}\PY{l+s+s1}{stR}\PY{l+s+s1}{\PYZsq{}}\PY{p}{;} \PY{n+nb}{print}\PY{p}{(}\PY{n}{r} \PY{o}{==} \PY{n}{t}\PY{p}{)}
\PY{n}{g} \PY{o}{=} \PY{l+s+s1}{\PYZsq{}}\PY{l+s+s1}{str}\PY{l+s+s1}{\PYZsq{}}\PY{p}{;} \PY{n}{x} \PY{o}{=} \PY{l+s+s1}{\PYZsq{}}\PY{l+s+s1}{str}\PY{l+s+s1}{\PYZsq{}}\PY{p}{;} \PY{n+nb}{print}\PY{p}{(}\PY{n}{g} \PY{o}{==} \PY{n}{x}\PY{p}{)}
\end{Verbatim}
\end{tcolorbox}

    \begin{Verbatim}[commandchars=\\\{\}]
True
False
True
    \end{Verbatim}

    \begin{enumerate}
\def\labelenumi{(\arabic{enumi})}
\setcounter{enumi}{5}
\tightlist
\item
  Not equal to: !=
\end{enumerate}

    \begin{tcolorbox}[breakable, size=fbox, boxrule=1pt, pad at break*=1mm,colback=cellbackground, colframe=cellborder]
\prompt{In}{incolor}{105}{\boxspacing}
\begin{Verbatim}[commandchars=\\\{\}]
\PY{n}{e} \PY{o}{=} \PY{l+m+mi}{2}\PY{p}{;} \PY{n}{s} \PY{o}{=} \PY{l+m+mi}{3}\PY{p}{;} \PY{n+nb}{print}\PY{p}{(}\PY{n}{e} \PY{o}{!=} \PY{n}{s}\PY{p}{)}
\end{Verbatim}
\end{tcolorbox}

    \begin{Verbatim}[commandchars=\\\{\}]
True
    \end{Verbatim}

    \begin{enumerate}
\def\labelenumi{(\arabic{enumi})}
\setcounter{enumi}{6}
\tightlist
\item
  Boolean Not: not
\end{enumerate}

    \begin{tcolorbox}[breakable, size=fbox, boxrule=1pt, pad at break*=1mm,colback=cellbackground, colframe=cellborder]
\prompt{In}{incolor}{106}{\boxspacing}
\begin{Verbatim}[commandchars=\\\{\}]
\PY{n}{u} \PY{o}{=} \PY{k+kc}{True}\PY{p}{;} \PY{n+nb}{print}\PY{p}{(}\PY{o+ow}{not} \PY{n}{u}\PY{p}{)}
\end{Verbatim}
\end{tcolorbox}

    \begin{Verbatim}[commandchars=\\\{\}]
False
    \end{Verbatim}

    \begin{enumerate}
\def\labelenumi{(\arabic{enumi})}
\setcounter{enumi}{7}
\tightlist
\item
  Boolean And: and
\end{enumerate}

    \begin{tcolorbox}[breakable, size=fbox, boxrule=1pt, pad at break*=1mm,colback=cellbackground, colframe=cellborder]
\prompt{In}{incolor}{107}{\boxspacing}
\begin{Verbatim}[commandchars=\\\{\}]
\PY{n}{f} \PY{o}{=} \PY{k+kc}{False}\PY{p}{;} \PY{n}{l} \PY{o}{=} \PY{k+kc}{True}\PY{p}{;} \PY{n+nb}{print}\PY{p}{(}\PY{n}{f} \PY{o+ow}{and} \PY{n}{l}\PY{p}{)}
\end{Verbatim}
\end{tcolorbox}

    \begin{Verbatim}[commandchars=\\\{\}]
False
    \end{Verbatim}

    \begin{enumerate}
\def\labelenumi{(\arabic{enumi})}
\setcounter{enumi}{8}
\tightlist
\item
  Boolean Or: or
\end{enumerate}

    \begin{tcolorbox}[breakable, size=fbox, boxrule=1pt, pad at break*=1mm,colback=cellbackground, colframe=cellborder]
\prompt{In}{incolor}{108}{\boxspacing}
\begin{Verbatim}[commandchars=\\\{\}]
\PY{n}{f} \PY{o}{=} \PY{k+kc}{False}\PY{p}{;} \PY{n}{l} \PY{o}{=} \PY{k+kc}{True}\PY{p}{;} \PY{n+nb}{print}\PY{p}{(}\PY{n}{f} \PY{o+ow}{or} \PY{n}{l}\PY{p}{)}
\end{Verbatim}
\end{tcolorbox}

    \begin{Verbatim}[commandchars=\\\{\}]
True
    \end{Verbatim}

    Part 2. Variables and Types

    To better calculate, we set variables as these numbers, so we can
utilize those numbers only by calling variables names)

    \begin{enumerate}
\def\labelenumi{\arabic{enumi}.}
\tightlist
\item
  Five different types of variables
\end{enumerate}

    (1st step: set variables)

(Notes: Literal Constants: contains both strings and numbers. 1) Numbers
contains both integer and float; 2) Strings is the sequence of
characters)

    \begin{enumerate}
\def\labelenumi{(\arabic{enumi})}
\tightlist
\item
  integer: int
\end{enumerate}

    \begin{tcolorbox}[breakable, size=fbox, boxrule=1pt, pad at break*=1mm,colback=cellbackground, colframe=cellborder]
\prompt{In}{incolor}{2}{\boxspacing}
\begin{Verbatim}[commandchars=\\\{\}]
\PY{n}{savings} \PY{o}{=} \PY{l+m+mi}{100}
\end{Verbatim}
\end{tcolorbox}

    \begin{enumerate}
\def\labelenumi{(\arabic{enumi})}
\setcounter{enumi}{1}
\tightlist
\item
  float: float (special: long, None)
\end{enumerate}

    \begin{tcolorbox}[breakable, size=fbox, boxrule=1pt, pad at break*=1mm,colback=cellbackground, colframe=cellborder]
\prompt{In}{incolor}{3}{\boxspacing}
\begin{Verbatim}[commandchars=\\\{\}]
\PY{n}{growth\PYZus{}multiplier} \PY{o}{=} \PY{l+m+mf}{1.1}
\end{Verbatim}
\end{tcolorbox}

    \begin{enumerate}
\def\labelenumi{(\arabic{enumi})}
\setcounter{enumi}{2}
\tightlist
\item
  string: str (special: unicode)
\end{enumerate}

    \begin{tcolorbox}[breakable, size=fbox, boxrule=1pt, pad at break*=1mm,colback=cellbackground, colframe=cellborder]
\prompt{In}{incolor}{4}{\boxspacing}
\begin{Verbatim}[commandchars=\\\{\}]
\PY{n}{desc} \PY{o}{=} \PY{l+s+s2}{\PYZdq{}}\PY{l+s+s2}{compound interest}\PY{l+s+s2}{\PYZdq{}}
\end{Verbatim}
\end{tcolorbox}

    \begin{enumerate}
\def\labelenumi{(\arabic{enumi})}
\setcounter{enumi}{3}
\tightlist
\item
  boolean: bool
\end{enumerate}

    \begin{tcolorbox}[breakable, size=fbox, boxrule=1pt, pad at break*=1mm,colback=cellbackground, colframe=cellborder]
\prompt{In}{incolor}{5}{\boxspacing}
\begin{Verbatim}[commandchars=\\\{\}]
\PY{n}{profitable} \PY{o}{=} \PY{k+kc}{True}
\end{Verbatim}
\end{tcolorbox}

    \begin{enumerate}
\def\labelenumi{(\arabic{enumi})}
\setcounter{enumi}{4}
\tightlist
\item
  date
\end{enumerate}

    \begin{tcolorbox}[breakable, size=fbox, boxrule=1pt, pad at break*=1mm,colback=cellbackground, colframe=cellborder]
\prompt{In}{incolor}{6}{\boxspacing}
\begin{Verbatim}[commandchars=\\\{\}]
\PY{c+c1}{\PYZsh{} (Notes: we can use type() function to check the variable\PYZsq{}s type)}
\PY{n+nb}{print}\PY{p}{(}\PY{n+nb}{type}\PY{p}{(}\PY{n}{profitable}\PY{p}{)}\PY{p}{)}
\end{Verbatim}
\end{tcolorbox}

    \begin{Verbatim}[commandchars=\\\{\}]
<class 'bool'>
    \end{Verbatim}

    \begin{enumerate}
\def\labelenumi{\arabic{enumi}.}
\setcounter{enumi}{1}
\tightlist
\item
  Calculation among variables
\end{enumerate}

    2nd step: calculate with variables we set)

    \begin{enumerate}
\def\labelenumi{(\arabic{enumi})}
\tightlist
\item
  among numericals (integer or float)
\end{enumerate}

    \begin{tcolorbox}[breakable, size=fbox, boxrule=1pt, pad at break*=1mm,colback=cellbackground, colframe=cellborder]
\prompt{In}{incolor}{10}{\boxspacing}
\begin{Verbatim}[commandchars=\\\{\}]
\PY{n}{result} \PY{o}{=} \PY{n}{savings} \PY{o}{*} \PY{p}{(}\PY{n}{growth\PYZus{}multiplier} \PY{o}{*}\PY{o}{*} \PY{l+m+mi}{7}\PY{p}{)}

\PY{n+nb}{print}\PY{p}{(}\PY{n}{result}\PY{p}{)}
\end{Verbatim}
\end{tcolorbox}

    \begin{Verbatim}[commandchars=\\\{\}]
194.87171000000012
    \end{Verbatim}

    \begin{enumerate}
\def\labelenumi{(\arabic{enumi})}
\setcounter{enumi}{1}
\tightlist
\item
  among strings
\end{enumerate}

    \begin{tcolorbox}[breakable, size=fbox, boxrule=1pt, pad at break*=1mm,colback=cellbackground, colframe=cellborder]
\prompt{In}{incolor}{11}{\boxspacing}
\begin{Verbatim}[commandchars=\\\{\}]
\PY{n}{doubledesc} \PY{o}{=} \PY{n}{desc} \PY{o}{+} \PY{n}{desc}

\PY{n+nb}{print}\PY{p}{(}\PY{n}{doubledesc}\PY{p}{)}
\end{Verbatim}
\end{tcolorbox}

    \begin{Verbatim}[commandchars=\\\{\}]
compound interestcompound interest
    \end{Verbatim}

    \begin{enumerate}
\def\labelenumi{\arabic{enumi}.}
\setcounter{enumi}{2}
\tightlist
\item
  Transformation between different types of variables
\end{enumerate}

    3rd step: change the type of variable

    \begin{enumerate}
\def\labelenumi{(\arabic{enumi})}
\tightlist
\item
  str()
\end{enumerate}

    \begin{itemize}
\tightlist
\item
  str( ): change the numeric(int/float) to str
\end{itemize}

    \begin{tcolorbox}[breakable, size=fbox, boxrule=1pt, pad at break*=1mm,colback=cellbackground, colframe=cellborder]
\prompt{In}{incolor}{12}{\boxspacing}
\begin{Verbatim}[commandchars=\\\{\}]
\PY{n+nb}{print}\PY{p}{(}\PY{l+s+s2}{\PYZdq{}}\PY{l+s+s2}{I started with \PYZdl{}}\PY{l+s+s2}{\PYZdq{}} \PY{o}{+} \PY{n+nb}{str}\PY{p}{(}\PY{n}{savings}\PY{p}{)} \PY{o}{+} \PY{l+s+s2}{\PYZdq{}}\PY{l+s+s2}{ and now have \PYZdl{}}\PY{l+s+s2}{\PYZdq{}} \PY{o}{+} \PY{n+nb}{str}\PY{p}{(}\PY{n}{result}\PY{p}{)} \PY{o}{+} \PY{l+s+s2}{\PYZdq{}}\PY{l+s+s2}{. Awesome!}\PY{l+s+s2}{\PYZdq{}}\PY{p}{)}

\PY{n+nb}{print}\PY{p}{(}\PY{l+s+s2}{\PYZdq{}}\PY{l+s+s2}{I started with \PYZdl{}}\PY{l+s+s2}{\PYZdq{}}\PY{p}{,} \PY{n}{savings}\PY{p}{,} \PY{l+s+s2}{\PYZdq{}}\PY{l+s+s2}{ and now have \PYZdl{}}\PY{l+s+s2}{\PYZdq{}}\PY{p}{,} \PY{n}{result}\PY{p}{,} \PY{l+s+s2}{\PYZdq{}}\PY{l+s+s2}{. Awesome!}\PY{l+s+s2}{\PYZdq{}}\PY{p}{)} 
\PY{c+c1}{\PYZsh{} (if don\PYZsq{}t want to change the mode, then use \PYZdq{},\PYZdq{} to replace \PYZdq{}+\PYZdq{})}
\end{Verbatim}
\end{tcolorbox}

    \begin{Verbatim}[commandchars=\\\{\}]
I started with \$100 and now have \$194.87171000000012. Awesome!
I started with \$ 100  and now have \$ 194.87171000000012 . Awesome!
    \end{Verbatim}

    \begin{enumerate}
\def\labelenumi{(\arabic{enumi})}
\setcounter{enumi}{1}
\tightlist
\item
  int() / float()
\end{enumerate}

    \begin{itemize}
\tightlist
\item
  int( ) / float( ): change the string to numeric(int/float)
\end{itemize}

    \begin{tcolorbox}[breakable, size=fbox, boxrule=1pt, pad at break*=1mm,colback=cellbackground, colframe=cellborder]
\prompt{In}{incolor}{13}{\boxspacing}
\begin{Verbatim}[commandchars=\\\{\}]
\PY{n}{pi\PYZus{}string} \PY{o}{=} \PY{l+s+s2}{\PYZdq{}}\PY{l+s+s2}{3.1415926}\PY{l+s+s2}{\PYZdq{}}

\PY{n}{pi\PYZus{}float} \PY{o}{=} \PY{n+nb}{float}\PY{p}{(}\PY{n}{pi\PYZus{}string}\PY{p}{)}

\PY{n+nb}{print}\PY{p}{(}\PY{n}{pi\PYZus{}float}\PY{p}{)}
\end{Verbatim}
\end{tcolorbox}

    \begin{Verbatim}[commandchars=\\\{\}]
3.1415926
    \end{Verbatim}

    Part 3. Control Flow Statement

    \begin{enumerate}
\def\labelenumi{\arabic{enumi}.}
\tightlist
\item
  three types of control flow statements
\end{enumerate}

    \begin{enumerate}
\def\labelenumi{(\arabic{enumi})}
\tightlist
\item
  `if' statement
\end{enumerate}

    \begin{tcolorbox}[breakable, size=fbox, boxrule=1pt, pad at break*=1mm,colback=cellbackground, colframe=cellborder]
\prompt{In}{incolor}{ }{\boxspacing}
\begin{Verbatim}[commandchars=\\\{\}]
\PY{n}{number} \PY{o}{=} \PY{l+m+mi}{23}
\PY{n}{guess} \PY{o}{=} \PY{n+nb}{int}\PY{p}{(}\PY{n+nb}{input}\PY{p}{(}\PY{l+s+s1}{\PYZsq{}}\PY{l+s+s1}{Enter an integer : }\PY{l+s+s1}{\PYZsq{}}\PY{p}{)}\PY{p}{)}

\PY{k}{if} \PY{n}{guess} \PY{o}{==} \PY{n}{number}\PY{p}{:}
    \PY{n+nb}{print}\PY{p}{(}\PY{l+s+s1}{\PYZsq{}}\PY{l+s+s1}{Congratulations, you guessed it.}\PY{l+s+s1}{\PYZsq{}}\PY{p}{)}
    \PY{n+nb}{print}\PY{p}{(}\PY{l+s+s1}{\PYZsq{}}\PY{l+s+s1}{(but you do not win any prized)}\PY{l+s+s1}{\PYZsq{}}\PY{p}{)}
\PY{k}{elif} \PY{n}{guess} \PY{o}{\PYZlt{}} \PY{n}{number}\PY{p}{:}
    \PY{n+nb}{print}\PY{p}{(}\PY{l+s+s1}{\PYZsq{}}\PY{l+s+s1}{No, it it a little higher than that.}\PY{l+s+s1}{\PYZsq{}}\PY{p}{)}
\PY{k}{else}\PY{p}{:}
    \PY{n+nb}{print}\PY{p}{(}\PY{l+s+s1}{\PYZsq{}}\PY{l+s+s1}{No, it is a little lower than that.}\PY{l+s+s1}{\PYZsq{}}\PY{p}{)}
\PY{n+nb}{print}\PY{p}{(}\PY{l+s+s1}{\PYZsq{}}\PY{l+s+s1}{Done}\PY{l+s+s1}{\PYZsq{}}\PY{p}{)}
\end{Verbatim}
\end{tcolorbox}

    \begin{enumerate}
\def\labelenumi{(\arabic{enumi})}
\setcounter{enumi}{1}
\tightlist
\item
  `while' statement
\end{enumerate}

    \begin{tcolorbox}[breakable, size=fbox, boxrule=1pt, pad at break*=1mm,colback=cellbackground, colframe=cellborder]
\prompt{In}{incolor}{ }{\boxspacing}
\begin{Verbatim}[commandchars=\\\{\}]
\PY{n}{number} \PY{o}{=} \PY{l+m+mi}{23}
\PY{n}{running} \PY{o}{=} \PY{k+kc}{True}

\PY{k}{while} \PY{n}{running}\PY{p}{:}
    \PY{n}{guess} \PY{o}{=} \PY{n+nb}{int}\PY{p}{(}\PY{n+nb}{input}\PY{p}{(}\PY{l+s+s1}{\PYZsq{}}\PY{l+s+s1}{Enter an integer : }\PY{l+s+s1}{\PYZsq{}}\PY{p}{)}\PY{p}{)}
    
    \PY{k}{if} \PY{n}{guess} \PY{o}{==} \PY{n}{number}\PY{p}{:}
        \PY{n+nb}{print}\PY{p}{(}\PY{l+s+s1}{\PYZsq{}}\PY{l+s+s1}{Congratulations, you guessed it.}\PY{l+s+s1}{\PYZsq{}}\PY{p}{)}
        \PY{n}{running} \PY{o}{=} \PY{k+kc}{False}
    \PY{k}{elif} \PY{n}{guess} \PY{o}{\PYZlt{}} \PY{n}{number}\PY{p}{:}
        \PY{n+nb}{print}\PY{p}{(}\PY{l+s+s1}{\PYZsq{}}\PY{l+s+s1}{No, it is a little higher than that.}\PY{l+s+s1}{\PYZsq{}}\PY{p}{)}
    \PY{k}{else}\PY{p}{:}
        \PY{n+nb}{print}\PY{p}{(}\PY{l+s+s1}{\PYZsq{}}\PY{l+s+s1}{No, it is a little lower than that.}\PY{l+s+s1}{\PYZsq{}}\PY{p}{)}
\PY{k}{else}\PY{p}{:}
    \PY{n+nb}{print}\PY{p}{(}\PY{l+s+s1}{\PYZsq{}}\PY{l+s+s1}{The while loop is over.}\PY{l+s+s1}{\PYZsq{}}\PY{p}{)}

\PY{n+nb}{print}\PY{p}{(}\PY{l+s+s1}{\PYZsq{}}\PY{l+s+s1}{Done}\PY{l+s+s1}{\PYZsq{}}\PY{p}{)}
\end{Verbatim}
\end{tcolorbox}

    \begin{enumerate}
\def\labelenumi{(\arabic{enumi})}
\setcounter{enumi}{2}
\tightlist
\item
  `for' statement
\end{enumerate}

    \begin{tcolorbox}[breakable, size=fbox, boxrule=1pt, pad at break*=1mm,colback=cellbackground, colframe=cellborder]
\prompt{In}{incolor}{11}{\boxspacing}
\begin{Verbatim}[commandchars=\\\{\}]
\PY{k}{for} \PY{n}{i} \PY{o+ow}{in} \PY{n+nb}{range}\PY{p}{(}\PY{l+m+mi}{1}\PY{p}{,}\PY{l+m+mi}{5}\PY{p}{)}\PY{p}{:}
    \PY{n+nb}{print}\PY{p}{(}\PY{n}{i}\PY{p}{)}
\PY{k}{else}\PY{p}{:}
    \PY{n+nb}{print}\PY{p}{(}\PY{l+s+s1}{\PYZsq{}}\PY{l+s+s1}{The for loop is over}\PY{l+s+s1}{\PYZsq{}}\PY{p}{)}
\end{Verbatim}
\end{tcolorbox}

    \begin{Verbatim}[commandchars=\\\{\}]
1
2
3
4
The for loop is over
    \end{Verbatim}

    \begin{enumerate}
\def\labelenumi{\arabic{enumi}.}
\setcounter{enumi}{1}
\tightlist
\item
  Three types of statements used in the three types of control flow
  statements above
\end{enumerate}

    There are three types of statements used in the three types of control
flow statements above, which are respectively:

\begin{itemize}
\item
  \begin{enumerate}
  \def\labelenumi{(\arabic{enumi})}
  \tightlist
  \item
    `break' statement
  \end{enumerate}
\item
  \begin{enumerate}
  \def\labelenumi{(\arabic{enumi})}
  \setcounter{enumi}{1}
  \tightlist
  \item
    `continue' statement
  \end{enumerate}
\item
  \begin{enumerate}
  \def\labelenumi{(\arabic{enumi})}
  \setcounter{enumi}{2}
  \tightlist
  \item
    `pass' statement
  \end{enumerate}
\end{itemize}

\textbf{Notes:} the difference between these three statements:

\begin{itemize}
\item
  \textbf{1)`continue' statement:} skips all the statements after itself
  (loop\_A i) and restart a new loop (loop\_A i+1), which means that we
  still use `while' or `for' loop but use a new number;
\item
  \textbf{2)`pass' statement:} is only used as a placeholder,statements
  after it in a loop can still be executed;
\item
  \textbf{3)`break' statement:} skips all the statements after itself
  (loop\_A i) and stop running this loop(entire loop\_A), which means
  that we just don't execute anything in the `while' or `for' loop)
\end{itemize}

    \begin{enumerate}
\def\labelenumi{(\arabic{enumi})}
\tightlist
\item
  `pass' statement
\end{enumerate}

    \begin{itemize}
\tightlist
\item
  `pass' statement: only placeholder, still execute statements in the
  loop
\end{itemize}

    \begin{tcolorbox}[breakable, size=fbox, boxrule=1pt, pad at break*=1mm,colback=cellbackground, colframe=cellborder]
\prompt{In}{incolor}{14}{\boxspacing}
\begin{Verbatim}[commandchars=\\\{\}]
\PY{n}{a} \PY{o}{=} \PY{l+s+s1}{\PYZsq{}}\PY{l+s+s1}{pythyon}\PY{l+s+s1}{\PYZsq{}}
\PY{n}{i} \PY{o}{=} \PY{l+m+mi}{2}
\PY{k}{for} \PY{n}{element} \PY{o+ow}{in} \PY{n}{a}\PY{p}{:}
    \PY{k}{if} \PY{n}{element} \PY{o}{==} \PY{l+s+s1}{\PYZsq{}}\PY{l+s+s1}{y}\PY{l+s+s1}{\PYZsq{}}\PY{p}{:}
        \PY{k}{pass}
        \PY{n}{i} \PY{o}{=} \PY{l+m+mi}{3}
    \PY{k}{else}\PY{p}{:}
        \PY{n+nb}{print}\PY{p}{(}\PY{n}{element}\PY{o}{+}\PY{n+nb}{str}\PY{p}{(}\PY{n}{i}\PY{p}{)}\PY{p}{)}
\end{Verbatim}
\end{tcolorbox}

    \begin{Verbatim}[commandchars=\\\{\}]
p2
t3
h3
o3
n3
    \end{Verbatim}

    \begin{enumerate}
\def\labelenumi{(\arabic{enumi})}
\setcounter{enumi}{1}
\tightlist
\item
  `continue' statement
\end{enumerate}

    \begin{itemize}
\tightlist
\item
  `continue' statement: stronger than `pass', but less stronger than
  `break', skip the following statements, and restart a new loop.
\end{itemize}

    \begin{tcolorbox}[breakable, size=fbox, boxrule=1pt, pad at break*=1mm,colback=cellbackground, colframe=cellborder]
\prompt{In}{incolor}{15}{\boxspacing}
\begin{Verbatim}[commandchars=\\\{\}]
\PY{c+c1}{\PYZsh{}   b). }
\PY{n}{a} \PY{o}{=} \PY{l+s+s1}{\PYZsq{}}\PY{l+s+s1}{pythyon}\PY{l+s+s1}{\PYZsq{}}
\PY{n}{i} \PY{o}{=} \PY{l+m+mi}{2}
\PY{k}{for} \PY{n}{element} \PY{o+ow}{in} \PY{n}{a}\PY{p}{:}
    \PY{k}{if} \PY{n}{element} \PY{o}{==} \PY{l+s+s1}{\PYZsq{}}\PY{l+s+s1}{y}\PY{l+s+s1}{\PYZsq{}}\PY{p}{:}
        \PY{k}{continue}
        \PY{n}{i} \PY{o}{=} \PY{l+m+mi}{3}
    \PY{k}{else}\PY{p}{:}
        \PY{n+nb}{print}\PY{p}{(}\PY{n}{element}\PY{o}{+}\PY{n+nb}{str}\PY{p}{(}\PY{n}{i}\PY{p}{)}\PY{p}{)}
\end{Verbatim}
\end{tcolorbox}

    \begin{Verbatim}[commandchars=\\\{\}]
p2
t2
h2
o2
n2
    \end{Verbatim}

    \begin{tcolorbox}[breakable, size=fbox, boxrule=1pt, pad at break*=1mm,colback=cellbackground, colframe=cellborder]
\prompt{In}{incolor}{18}{\boxspacing}
\begin{Verbatim}[commandchars=\\\{\}]
\PY{c+c1}{\PYZsh{}  another example of\PYZsq{}continue\PYZsq{} statement}
\PY{k}{while} \PY{k+kc}{True}\PY{p}{:}
    \PY{n}{s} \PY{o}{=} \PY{n+nb}{input}\PY{p}{(}\PY{l+s+s1}{\PYZsq{}}\PY{l+s+s1}{Enter something : }\PY{l+s+s1}{\PYZsq{}}\PY{p}{)}
    \PY{k}{if} \PY{n}{s} \PY{o}{==} \PY{l+s+s1}{\PYZsq{}}\PY{l+s+s1}{quit}\PY{l+s+s1}{\PYZsq{}}\PY{p}{:}
        \PY{k}{break}
    \PY{k}{if} \PY{n+nb}{len}\PY{p}{(}\PY{n}{s}\PY{p}{)} \PY{o}{\PYZlt{}} \PY{l+m+mi}{3}\PY{p}{:}
        \PY{n+nb}{print}\PY{p}{(}\PY{l+s+s1}{\PYZsq{}}\PY{l+s+s1}{Too small}\PY{l+s+s1}{\PYZsq{}}\PY{p}{)}
        \PY{k}{continue}
    \PY{n+nb}{print}\PY{p}{(}\PY{l+s+s1}{\PYZsq{}}\PY{l+s+s1}{Input is of sufficient length}\PY{l+s+s1}{\PYZsq{}}\PY{p}{)}
\PY{c+c1}{\PYZsh{} (Notes: We can use \PYZsq{}Continue\PYZsq{} to skip the remaining statements, or the remaining statments will be executed.)}
\PY{c+c1}{\PYZsh{} how to understand??? skip how many steps???}
\end{Verbatim}
\end{tcolorbox}

    \begin{Verbatim}[commandchars=\\\{\}]
Enter something : quit
    \end{Verbatim}

    \begin{enumerate}
\def\labelenumi{(\arabic{enumi})}
\setcounter{enumi}{2}
\tightlist
\item
  `break' statement
\end{enumerate}

    \begin{itemize}
\tightlist
\item
  `break' statement: most strongest statements, stop running the loop
\end{itemize}

    \begin{tcolorbox}[breakable, size=fbox, boxrule=1pt, pad at break*=1mm,colback=cellbackground, colframe=cellborder]
\prompt{In}{incolor}{19}{\boxspacing}
\begin{Verbatim}[commandchars=\\\{\}]
\PY{n}{a} \PY{o}{=} \PY{l+s+s1}{\PYZsq{}}\PY{l+s+s1}{pythyon}\PY{l+s+s1}{\PYZsq{}}
\PY{n}{i} \PY{o}{=} \PY{l+m+mi}{2}
\PY{k}{for} \PY{n}{element} \PY{o+ow}{in} \PY{n}{a}\PY{p}{:}
    \PY{k}{if} \PY{n}{element} \PY{o}{==} \PY{l+s+s1}{\PYZsq{}}\PY{l+s+s1}{y}\PY{l+s+s1}{\PYZsq{}}\PY{p}{:}
        \PY{k}{break}
        \PY{n}{i} \PY{o}{=} \PY{l+m+mi}{3}
    \PY{k}{else}\PY{p}{:}
        \PY{n+nb}{print}\PY{p}{(}\PY{n}{element}\PY{o}{+}\PY{n+nb}{str}\PY{p}{(}\PY{n}{i}\PY{p}{)}\PY{p}{)} 
\end{Verbatim}
\end{tcolorbox}

    \begin{Verbatim}[commandchars=\\\{\}]
p2
    \end{Verbatim}

    \begin{tcolorbox}[breakable, size=fbox, boxrule=1pt, pad at break*=1mm,colback=cellbackground, colframe=cellborder]
\prompt{In}{incolor}{20}{\boxspacing}
\begin{Verbatim}[commandchars=\\\{\}]
\PY{c+c1}{\PYZsh{}  another example of\PYZsq{}break\PYZsq{} statement}
\PY{k}{while} \PY{k+kc}{True}\PY{p}{:} \PY{c+c1}{\PYZsh{} how to understand \PYZsq{}True\PYZsq{} here???}
    \PY{n}{s} \PY{o}{=} \PY{n+nb}{input}\PY{p}{(}\PY{l+s+s1}{\PYZsq{}}\PY{l+s+s1}{Enter something : }\PY{l+s+s1}{\PYZsq{}}\PY{p}{)}
    \PY{k}{if} \PY{n}{s} \PY{o}{==} \PY{l+s+s1}{\PYZsq{}}\PY{l+s+s1}{quit}\PY{l+s+s1}{\PYZsq{}}\PY{p}{:}
        \PY{k}{break}
    \PY{n+nb}{print}\PY{p}{(}\PY{l+s+s1}{\PYZsq{}}\PY{l+s+s1}{Length of the string is}\PY{l+s+s1}{\PYZsq{}}\PY{p}{,} \PY{n+nb}{len}\PY{p}{(}\PY{n}{s}\PY{p}{)}\PY{p}{)}
\PY{n+nb}{print}\PY{p}{(}\PY{l+s+s1}{\PYZsq{}}\PY{l+s+s1}{Done}\PY{l+s+s1}{\PYZsq{}}\PY{p}{)}
\PY{c+c1}{\PYZsh{} (Notes: if you break one \PYZsq{}for\PYZsq{} / \PYZsq{}while\PYZsq{} statement, all the \PYZsq{}else\PYZsq{} module will not be executed.)}

\PY{c+c1}{\PYZsh{} ???Difference between break and end???}
\end{Verbatim}
\end{tcolorbox}

    \begin{Verbatim}[commandchars=\\\{\}]
Enter something : quit
Done
    \end{Verbatim}

    \begin{enumerate}
\def\labelenumi{(\arabic{enumi})}
\setcounter{enumi}{2}
\tightlist
\item
  `return' statement (*)
\end{enumerate}

    \textbf{Note:} If there is nothing behind return, that means `return
None'

??what's the difference between `return' and `print' (return just turn
back to one value, but no need to print it our)

    \begin{tcolorbox}[breakable, size=fbox, boxrule=1pt, pad at break*=1mm,colback=cellbackground, colframe=cellborder]
\prompt{In}{incolor}{21}{\boxspacing}
\begin{Verbatim}[commandchars=\\\{\}]
\PY{c+c1}{\PYZsh{} (2). \PYZsq{}return\PYZsq{} statement}
\PY{k}{def} \PY{n+nf}{maximum}\PY{p}{(}\PY{n}{x}\PY{p}{,} \PY{n}{y}\PY{p}{)}\PY{p}{:}
    \PY{k}{if} \PY{n}{x} \PY{o}{\PYZgt{}} \PY{n}{y}\PY{p}{:}
        \PY{k}{return} \PY{n}{x}
    \PY{k}{elif} \PY{n}{x} \PY{o}{==} \PY{n}{y}\PY{p}{:}
        \PY{k}{return} \PY{l+s+s1}{\PYZsq{}}\PY{l+s+s1}{The numbers are equal}\PY{l+s+s1}{\PYZsq{}}
    \PY{k}{else}\PY{p}{:}
        \PY{k}{return} \PY{n}{y}
\PY{n+nb}{print}\PY{p}{(}\PY{n}{maximum}\PY{p}{(}\PY{l+m+mi}{2}\PY{p}{,} \PY{l+m+mi}{3}\PY{p}{)}\PY{p}{)}
\end{Verbatim}
\end{tcolorbox}

    \begin{Verbatim}[commandchars=\\\{\}]
3
    \end{Verbatim}

    Part 4. Functions

    Definition: function is a slice of codes which can be used repeatedly.
You can give this function a specific name, and call this function at
any places via its name

    \begin{enumerate}
\def\labelenumi{\arabic{enumi}.}
\tightlist
\item
  Basic Approach to use function
\end{enumerate}

    \begin{enumerate}
\def\labelenumi{(\arabic{enumi})}
\tightlist
\item
  function without parameters
\end{enumerate}

    \begin{tcolorbox}[breakable, size=fbox, boxrule=1pt, pad at break*=1mm,colback=cellbackground, colframe=cellborder]
\prompt{In}{incolor}{23}{\boxspacing}
\begin{Verbatim}[commandchars=\\\{\}]
\PY{c+c1}{\PYZsh{}  1) step1: define a function}
\PY{k}{def} \PY{n+nf}{say\PYZus{}hello}\PY{p}{(}\PY{p}{)}\PY{p}{:} \PY{c+c1}{\PYZsh{} you can put parameters in (), which is mentioned as below}
    \PY{n+nb}{print}\PY{p}{(}\PY{l+s+s1}{\PYZsq{}}\PY{l+s+s1}{hello world}\PY{l+s+s1}{\PYZsq{}}\PY{p}{)}
\end{Verbatim}
\end{tcolorbox}

    \begin{Verbatim}[commandchars=\\\{\}]
hello world
    \end{Verbatim}

    \begin{tcolorbox}[breakable, size=fbox, boxrule=1pt, pad at break*=1mm,colback=cellbackground, colframe=cellborder]
\prompt{In}{incolor}{ }{\boxspacing}
\begin{Verbatim}[commandchars=\\\{\}]
\PY{c+c1}{\PYZsh{}  2) step2: call this function}
\PY{n}{say\PYZus{}hello}\PY{p}{(}\PY{p}{)}
\end{Verbatim}
\end{tcolorbox}

    \begin{enumerate}
\def\labelenumi{(\arabic{enumi})}
\setcounter{enumi}{1}
\tightlist
\item
  function with parameters
\end{enumerate}

    \begin{tcolorbox}[breakable, size=fbox, boxrule=1pt, pad at break*=1mm,colback=cellbackground, colframe=cellborder]
\prompt{In}{incolor}{24}{\boxspacing}
\begin{Verbatim}[commandchars=\\\{\}]
\PY{c+c1}{\PYZsh{}  1) step1: define a function}
\PY{k}{def} \PY{n+nf}{print\PYZus{}max}\PY{p}{(}\PY{n}{a}\PY{p}{,} \PY{n}{b}\PY{p}{)}\PY{p}{:}
    \PY{k}{if} \PY{n}{a} \PY{o}{\PYZgt{}} \PY{n}{b}\PY{p}{:}
        \PY{n+nb}{print}\PY{p}{(}\PY{n}{a}\PY{p}{,} \PY{l+s+s1}{\PYZsq{}}\PY{l+s+s1}{is maximum}\PY{l+s+s1}{\PYZsq{}}\PY{p}{)}
    \PY{k}{elif} \PY{n}{a} \PY{o}{==} \PY{n}{b}\PY{p}{:}
        \PY{n+nb}{print}\PY{p}{(}\PY{n}{a}\PY{p}{,} \PY{l+s+s1}{\PYZsq{}}\PY{l+s+s1}{is equal to}\PY{l+s+s1}{\PYZsq{}}\PY{p}{,} \PY{n}{b}\PY{p}{)}
    \PY{k}{else}\PY{p}{:}
        \PY{n+nb}{print}\PY{p}{(}\PY{n}{b}\PY{p}{,} \PY{l+s+s1}{\PYZsq{}}\PY{l+s+s1}{is maximum}\PY{l+s+s1}{\PYZsq{}}\PY{p}{)}
\end{Verbatim}
\end{tcolorbox}

    \begin{Verbatim}[commandchars=\\\{\}]
4 is maximum
7 is maximum
    \end{Verbatim}

    \begin{tcolorbox}[breakable, size=fbox, boxrule=1pt, pad at break*=1mm,colback=cellbackground, colframe=cellborder]
\prompt{In}{incolor}{ }{\boxspacing}
\begin{Verbatim}[commandchars=\\\{\}]
\PY{c+c1}{\PYZsh{}  2) step2: call this function}
\PY{n}{print\PYZus{}max}\PY{p}{(}\PY{l+m+mi}{3}\PY{p}{,} \PY{l+m+mi}{4}\PY{p}{)}

\PY{c+c1}{\PYZsh{}  or use parameters to transfer variables}
\PY{n}{x} \PY{o}{=} \PY{l+m+mi}{5}
\PY{n}{y} \PY{o}{=} \PY{l+m+mi}{7}
\PY{n}{print\PYZus{}max}\PY{p}{(}\PY{l+m+mi}{5}\PY{p}{,}\PY{l+m+mi}{7}\PY{p}{)}
\end{Verbatim}
\end{tcolorbox}

    \begin{enumerate}
\def\labelenumi{\arabic{enumi}.}
\setcounter{enumi}{1}
\tightlist
\item
  Local Variables \& Global Variables
\end{enumerate}

    \begin{enumerate}
\def\labelenumi{(\arabic{enumi})}
\tightlist
\item
  Local Variables
\end{enumerate}

    \begin{tcolorbox}[breakable, size=fbox, boxrule=1pt, pad at break*=1mm,colback=cellbackground, colframe=cellborder]
\prompt{In}{incolor}{25}{\boxspacing}
\begin{Verbatim}[commandchars=\\\{\}]
\PY{n}{x} \PY{o}{=} \PY{l+m+mi}{50}

\PY{c+c1}{\PYZsh{}  1) step1: define a function}
\PY{k}{def} \PY{n+nf}{func}\PY{p}{(}\PY{n}{x}\PY{p}{)}\PY{p}{:}
    \PY{n+nb}{print}\PY{p}{(}\PY{l+s+s1}{\PYZsq{}}\PY{l+s+s1}{x is}\PY{l+s+s1}{\PYZsq{}}\PY{p}{,} \PY{n}{x}\PY{p}{)}
    \PY{n}{x} \PY{o}{=} \PY{l+m+mi}{2}
    \PY{n+nb}{print}\PY{p}{(}\PY{l+s+s1}{\PYZsq{}}\PY{l+s+s1}{Changed local x to}\PY{l+s+s1}{\PYZsq{}}\PY{p}{,} \PY{n}{x}\PY{p}{)}
\end{Verbatim}
\end{tcolorbox}

    \begin{Verbatim}[commandchars=\\\{\}]
x is 50
Changed local x to 2
x is still 50
    \end{Verbatim}

    \begin{tcolorbox}[breakable, size=fbox, boxrule=1pt, pad at break*=1mm,colback=cellbackground, colframe=cellborder]
\prompt{In}{incolor}{ }{\boxspacing}
\begin{Verbatim}[commandchars=\\\{\}]
\PY{c+c1}{\PYZsh{}  2) step2: call this function}
\PY{n}{func}\PY{p}{(}\PY{n}{x}\PY{p}{)}

\PY{n+nb}{print}\PY{p}{(}\PY{l+s+s1}{\PYZsq{}}\PY{l+s+s1}{x is still}\PY{l+s+s1}{\PYZsq{}}\PY{p}{,} \PY{n}{x}\PY{p}{)} \PY{c+c1}{\PYZsh{} function doesn\PYZsq{}t change the value of x}
\end{Verbatim}
\end{tcolorbox}

    \begin{enumerate}
\def\labelenumi{(\arabic{enumi})}
\tightlist
\item
  Global Variables
\end{enumerate}

    \begin{tcolorbox}[breakable, size=fbox, boxrule=1pt, pad at break*=1mm,colback=cellbackground, colframe=cellborder]
\prompt{In}{incolor}{26}{\boxspacing}
\begin{Verbatim}[commandchars=\\\{\}]
\PY{n}{x} \PY{o}{=} \PY{l+m+mi}{50}

\PY{c+c1}{\PYZsh{}  1) step1: define a function}
\PY{k}{def} \PY{n+nf}{func}\PY{p}{(}\PY{p}{)}\PY{p}{:} \PY{c+c1}{\PYZsh{}??? why there is no x in () in this function???}
    \PY{k}{global} \PY{n}{x} \PY{c+c1}{\PYZsh{} you can declare many variables to be global (like: global x, y, z)}
    
    \PY{n+nb}{print}\PY{p}{(}\PY{l+s+s1}{\PYZsq{}}\PY{l+s+s1}{x is}\PY{l+s+s1}{\PYZsq{}}\PY{p}{,} \PY{n}{x}\PY{p}{)}
    \PY{n}{x} \PY{o}{=} \PY{l+m+mi}{2}
    \PY{n+nb}{print}\PY{p}{(}\PY{l+s+s1}{\PYZsq{}}\PY{l+s+s1}{Changed global x to}\PY{l+s+s1}{\PYZsq{}}\PY{p}{,} \PY{n}{x}\PY{p}{)}
\end{Verbatim}
\end{tcolorbox}

    \begin{Verbatim}[commandchars=\\\{\}]
x is 50
Changed global x to 2
Value of x is 2
    \end{Verbatim}

    \begin{tcolorbox}[breakable, size=fbox, boxrule=1pt, pad at break*=1mm,colback=cellbackground, colframe=cellborder]
\prompt{In}{incolor}{ }{\boxspacing}
\begin{Verbatim}[commandchars=\\\{\}]
\PY{c+c1}{\PYZsh{}  2) step2: call this function}
\PY{n}{func}\PY{p}{(}\PY{p}{)}

\PY{n+nb}{print}\PY{p}{(}\PY{l+s+s1}{\PYZsq{}}\PY{l+s+s1}{Value of x is}\PY{l+s+s1}{\PYZsq{}}\PY{p}{,} \PY{n}{x}\PY{p}{)} \PY{c+c1}{\PYZsh{} function changes the value of x}
\end{Verbatim}
\end{tcolorbox}

    \begin{enumerate}
\def\labelenumi{\arabic{enumi}.}
\setcounter{enumi}{2}
\tightlist
\item
  Default value \& Keyword Arguments
\end{enumerate}

    \begin{enumerate}
\def\labelenumi{(\arabic{enumi})}
\tightlist
\item
  Default value
\end{enumerate}

    \textbf{Notes:} * 1) only the last variables of the function can set
default value; * 2) the variables with default value can \textbf{not} be
located before the variables without default value, which means that you
can only define `say(message, times = 1)', but \textbf{cannot} define
`say(times = 1, message)')

    \begin{tcolorbox}[breakable, size=fbox, boxrule=1pt, pad at break*=1mm,colback=cellbackground, colframe=cellborder]
\prompt{In}{incolor}{40}{\boxspacing}
\begin{Verbatim}[commandchars=\\\{\}]
\PY{c+c1}{\PYZsh{}  1) step1: define a function}
\PY{k}{def} \PY{n+nf}{say}\PY{p}{(}\PY{n}{message}\PY{p}{,} \PY{n}{times} \PY{o}{=} \PY{l+m+mi}{1}\PY{p}{)}\PY{p}{:}
    \PY{n+nb}{print}\PY{p}{(}\PY{n}{message} \PY{o}{*} \PY{n}{times}\PY{p}{)}
\end{Verbatim}
\end{tcolorbox}

    \begin{tcolorbox}[breakable, size=fbox, boxrule=1pt, pad at break*=1mm,colback=cellbackground, colframe=cellborder]
\prompt{In}{incolor}{42}{\boxspacing}
\begin{Verbatim}[commandchars=\\\{\}]
\PY{c+c1}{\PYZsh{}  2) step2: call this function}
\PY{c+c1}{\PYZsh{} Not change the default value}
\PY{n+nb}{print}\PY{p}{(}\PY{n}{say}\PY{p}{(}\PY{l+s+s1}{\PYZsq{}}\PY{l+s+s1}{Hello}\PY{l+s+s1}{\PYZsq{}}\PY{p}{)}\PY{p}{)}

\PY{c+c1}{\PYZsh{} change the default value}
\PY{n+nb}{print}\PY{p}{(}\PY{n}{say}\PY{p}{(}\PY{l+s+s1}{\PYZsq{}}\PY{l+s+s1}{world}\PY{l+s+s1}{\PYZsq{}}\PY{p}{,} \PY{l+m+mi}{5}\PY{p}{)}\PY{p}{)}
\end{Verbatim}
\end{tcolorbox}

    \begin{Verbatim}[commandchars=\\\{\}]
Hello
None
worldworldworldworldworld
None
    \end{Verbatim}

    \begin{enumerate}
\def\labelenumi{(\arabic{enumi})}
\tightlist
\item
  Keyword Arguments
\end{enumerate}

    \begin{tcolorbox}[breakable, size=fbox, boxrule=1pt, pad at break*=1mm,colback=cellbackground, colframe=cellborder]
\prompt{In}{incolor}{38}{\boxspacing}
\begin{Verbatim}[commandchars=\\\{\}]
\PY{c+c1}{\PYZsh{}  1) step1: define a function}
\PY{k}{def} \PY{n+nf}{func}\PY{p}{(}\PY{n}{a}\PY{p}{,} \PY{n}{b} \PY{o}{=} \PY{l+m+mi}{5}\PY{p}{,} \PY{n}{c} \PY{o}{=} \PY{l+m+mi}{10}\PY{p}{)}\PY{p}{:}
    \PY{n+nb}{print}\PY{p}{(}\PY{l+s+s1}{\PYZsq{}}\PY{l+s+s1}{a is}\PY{l+s+s1}{\PYZsq{}}\PY{p}{,} \PY{n}{a}\PY{p}{,} \PY{l+s+s1}{\PYZsq{}}\PY{l+s+s1}{and b is}\PY{l+s+s1}{\PYZsq{}}\PY{p}{,} \PY{n}{b}\PY{p}{,} \PY{l+s+s1}{\PYZsq{}}\PY{l+s+s1}{and c is}\PY{l+s+s1}{\PYZsq{}}\PY{p}{,} \PY{n}{c}\PY{p}{)}
\end{Verbatim}
\end{tcolorbox}

    \begin{tcolorbox}[breakable, size=fbox, boxrule=1pt, pad at break*=1mm,colback=cellbackground, colframe=cellborder]
\prompt{In}{incolor}{39}{\boxspacing}
\begin{Verbatim}[commandchars=\\\{\}]
\PY{c+c1}{\PYZsh{}  2) step2: call this function}
\PY{n+nb}{print}\PY{p}{(}\PY{n}{func}\PY{p}{(}\PY{l+m+mi}{3}\PY{p}{,} \PY{l+m+mi}{7}\PY{p}{)}\PY{p}{)}
\PY{n+nb}{print}\PY{p}{(}\PY{n}{func}\PY{p}{(}\PY{l+m+mi}{25}\PY{p}{,} \PY{n}{c} \PY{o}{=} \PY{l+m+mi}{24}\PY{p}{)}\PY{p}{)}
\PY{n+nb}{print}\PY{p}{(}\PY{n}{func}\PY{p}{(}\PY{n}{c} \PY{o}{=} \PY{l+m+mi}{50}\PY{p}{,} \PY{n}{a} \PY{o}{=} \PY{l+m+mi}{100}\PY{p}{)}\PY{p}{)} 
\end{Verbatim}
\end{tcolorbox}

    \begin{Verbatim}[commandchars=\\\{\}]
a is 3 and b is 7 and c is 10
None
a is 25 and b is 5 and c is 24
None
a is 100 and b is 5 and c is 50
None
    \end{Verbatim}

    \begin{enumerate}
\def\labelenumi{\arabic{enumi}.}
\setcounter{enumi}{3}
\tightlist
\item
  Changeable Parameters
\end{enumerate}

    Use star sign to represent.

\begin{itemize}
\tightlist
\item
  one star means one tuple
\item
  two stars means one dictionary
\end{itemize}

    \begin{tcolorbox}[breakable, size=fbox, boxrule=1pt, pad at break*=1mm,colback=cellbackground, colframe=cellborder]
\prompt{In}{incolor}{36}{\boxspacing}
\begin{Verbatim}[commandchars=\\\{\}]
\PY{c+c1}{\PYZsh{}  1) step1: define a function}
\PY{k}{def} \PY{n+nf}{total}\PY{p}{(}\PY{n}{a} \PY{o}{=} \PY{l+m+mi}{5}\PY{p}{,} \PY{o}{*}\PY{n}{numbers}\PY{p}{,} \PY{o}{*}\PY{o}{*}\PY{n}{phonebook}\PY{p}{)}\PY{p}{:}
    \PY{n+nb}{print}\PY{p}{(}\PY{l+s+s1}{\PYZsq{}}\PY{l+s+s1}{a}\PY{l+s+s1}{\PYZsq{}}\PY{p}{,} \PY{n}{a}\PY{p}{)}
    
    \PY{k}{for} \PY{n}{single\PYZus{}item} \PY{o+ow}{in} \PY{n}{numbers}\PY{p}{:}
        \PY{n+nb}{print}\PY{p}{(}\PY{l+s+s1}{\PYZsq{}}\PY{l+s+s1}{single\PYZus{}item}\PY{l+s+s1}{\PYZsq{}}\PY{p}{,} \PY{n}{single\PYZus{}item}\PY{p}{)}
    
    \PY{k}{for} \PY{n}{first\PYZus{}part}\PY{p}{,} \PY{n}{second\PYZus{}part} \PY{o+ow}{in} \PY{n}{phonebook}\PY{o}{.}\PY{n}{items}\PY{p}{(}\PY{p}{)}\PY{p}{:}
        \PY{n+nb}{print}\PY{p}{(}\PY{n}{first\PYZus{}part}\PY{p}{,} \PY{n}{second\PYZus{}part}\PY{p}{)}
\end{Verbatim}
\end{tcolorbox}

    \begin{tcolorbox}[breakable, size=fbox, boxrule=1pt, pad at break*=1mm,colback=cellbackground, colframe=cellborder]
\prompt{In}{incolor}{37}{\boxspacing}
\begin{Verbatim}[commandchars=\\\{\}]
\PY{c+c1}{\PYZsh{}  2) step2: call this function}
\PY{n+nb}{print}\PY{p}{(}\PY{n}{total}\PY{p}{(}\PY{l+m+mi}{10}\PY{p}{,} \PY{l+m+mi}{1}\PY{p}{,} \PY{l+m+mi}{2}\PY{p}{,} \PY{l+m+mi}{3}\PY{p}{,} \PY{n}{Jack} \PY{o}{=} \PY{l+m+mi}{1123}\PY{p}{,} \PY{n}{John} \PY{o}{=} \PY{l+m+mi}{2231}\PY{p}{,} \PY{n}{Inge} \PY{o}{=} \PY{l+m+mi}{1560}\PY{p}{)}\PY{p}{)}
\end{Verbatim}
\end{tcolorbox}

    \begin{Verbatim}[commandchars=\\\{\}]
a 10
single\_item 1
single\_item 2
single\_item 3
Jack 1123
John 2231
Inge 1560
None
    \end{Verbatim}

    \begin{enumerate}
\def\labelenumi{\arabic{enumi}.}
\setcounter{enumi}{4}
\tightlist
\item
  Documentation Strings (DocStrings)
\end{enumerate}

    \begin{tcolorbox}[breakable, size=fbox, boxrule=1pt, pad at break*=1mm,colback=cellbackground, colframe=cellborder]
\prompt{In}{incolor}{30}{\boxspacing}
\begin{Verbatim}[commandchars=\\\{\}]
\PY{c+c1}{\PYZsh{}  1) step1: define a function}
\PY{k}{def} \PY{n+nf}{print\PYZus{}max}\PY{p}{(}\PY{n}{x}\PY{p}{,} \PY{n}{y}\PY{p}{)}\PY{p}{:}
    \PY{l+s+sd}{\PYZsq{}\PYZsq{}\PYZsq{}Print the maximum of two numbers.}
\PY{l+s+sd}{    The two values must be integers.\PYZsq{}\PYZsq{}\PYZsq{}}
    \PY{n}{x} \PY{o}{=} \PY{n+nb}{int}\PY{p}{(}\PY{n}{x}\PY{p}{)}
    \PY{n}{y} \PY{o}{=} \PY{n+nb}{int}\PY{p}{(}\PY{n}{y}\PY{p}{)}
    
    \PY{k}{if} \PY{n}{x} \PY{o}{\PYZgt{}} \PY{n}{y}\PY{p}{:}
        \PY{n+nb}{print}\PY{p}{(}\PY{n}{x}\PY{p}{,} \PY{l+s+s1}{\PYZsq{}}\PY{l+s+s1}{is maximum}\PY{l+s+s1}{\PYZsq{}}\PY{p}{)}
    \PY{k}{else}\PY{p}{:}
        \PY{n+nb}{print}\PY{p}{(}\PY{n}{y}\PY{p}{,} \PY{l+s+s1}{\PYZsq{}}\PY{l+s+s1}{is maximum}\PY{l+s+s1}{\PYZsq{}}\PY{p}{)}
\end{Verbatim}
\end{tcolorbox}

    \begin{Verbatim}[commandchars=\\\{\}]
5 is maximum
Print the maximum of two numbers.
    The two values must be integers.
    \end{Verbatim}

    \begin{tcolorbox}[breakable, size=fbox, boxrule=1pt, pad at break*=1mm,colback=cellbackground, colframe=cellborder]
\prompt{In}{incolor}{ }{\boxspacing}
\begin{Verbatim}[commandchars=\\\{\}]
\PY{n}{print\PYZus{}max}\PY{p}{(}\PY{l+m+mi}{3}\PY{p}{,} \PY{l+m+mi}{5}\PY{p}{)}
\PY{n+nb}{print}\PY{p}{(}\PY{n}{print\PYZus{}max}\PY{o}{.}\PY{n+nv+vm}{\PYZus{}\PYZus{}doc\PYZus{}\PYZus{}}\PY{p}{)} \PY{c+c1}{\PYZsh{} Note: here has two \PYZsq{}\PYZus{}\PYZsq{} !!!}
\end{Verbatim}
\end{tcolorbox}

    Part 5. Modules \& Packages

    This part is kind of difficult, what is exactly the module? What the
differenct between modules and packages? What's the applications?

It seems like module is just a ``.py'' file, what does this mean?

https://www.runoob.com/python/python-object.html

    \begin{enumerate}
\def\labelenumi{\arabic{enumi}.}
\tightlist
\item
  Modules
\end{enumerate}

    \begin{tcolorbox}[breakable, size=fbox, boxrule=1pt, pad at break*=1mm,colback=cellbackground, colframe=cellborder]
\prompt{In}{incolor}{31}{\boxspacing}
\begin{Verbatim}[commandchars=\\\{\}]
\PY{c+c1}{\PYZsh{} (1) common import approach}

\PY{k+kn}{import} \PY{n+nn}{sys}

\PY{n+nb}{print}\PY{p}{(}\PY{l+s+s1}{\PYZsq{}}\PY{l+s+s1}{The command line arguments are:}\PY{l+s+s1}{\PYZsq{}}\PY{p}{)}
\PY{k}{for} \PY{n}{i} \PY{o+ow}{in} \PY{n}{sys}\PY{o}{.}\PY{n}{argv}\PY{p}{:}
    \PY{n+nb}{print}\PY{p}{(}\PY{n}{i}\PY{p}{)}
    
\PY{n+nb}{print}\PY{p}{(}\PY{l+s+s1}{\PYZsq{}}\PY{l+s+se}{\PYZbs{}n}\PY{l+s+se}{\PYZbs{}n}\PY{l+s+s1}{The PYTHONPATH is}\PY{l+s+s1}{\PYZsq{}}\PY{p}{,} \PY{n}{sys}\PY{o}{.}\PY{n}{path}\PY{p}{,} \PY{l+s+s1}{\PYZsq{}}\PY{l+s+se}{\PYZbs{}n}\PY{l+s+s1}{\PYZsq{}}\PY{p}{)}
\PY{c+c1}{\PYZsh{}??? Don\PYZsq{}t understand the meaning here!}
\end{Verbatim}
\end{tcolorbox}

    \begin{Verbatim}[commandchars=\\\{\}]
The command line arguments are:
/Users/qingyuesu/opt/anaconda3/lib/python3.8/site-packages/ipykernel\_launcher.py
-f
/Users/qingyuesu/Library/Jupyter/runtime/kernel-6c151aef-6313-476e-80af-9589b59d
9af6.json


The PYTHONPATH is ['/Users/qingyuesu',
'/Users/qingyuesu/opt/anaconda3/lib/python38.zip',
'/Users/qingyuesu/opt/anaconda3/lib/python3.8',
'/Users/qingyuesu/opt/anaconda3/lib/python3.8/lib-dynload', '',
'/Users/qingyuesu/opt/anaconda3/lib/python3.8/site-packages',
'/Users/qingyuesu/opt/anaconda3/lib/python3.8/site-packages/aeosa',
'/Users/qingyuesu/opt/anaconda3/lib/python3.8/site-packages/IPython/extensions',
'/Users/qingyuesu/.ipython']

    \end{Verbatim}

    \begin{tcolorbox}[breakable, size=fbox, boxrule=1pt, pad at break*=1mm,colback=cellbackground, colframe=cellborder]
\prompt{In}{incolor}{32}{\boxspacing}
\begin{Verbatim}[commandchars=\\\{\}]
\PY{c+c1}{\PYZsh{} (2) from ... import ...}

\PY{k+kn}{from} \PY{n+nn}{math} \PY{k+kn}{import} \PY{n}{sqrt}
\PY{n+nb}{print}\PY{p}{(}\PY{l+s+s1}{\PYZsq{}}\PY{l+s+s1}{Square root of 16 is}\PY{l+s+s1}{\PYZsq{}}\PY{p}{,} \PY{n}{sqrt}\PY{p}{(}\PY{l+m+mi}{16}\PY{p}{)}\PY{p}{)}
\PY{c+c1}{\PYZsh{} the benefit of this method is that you can use \PYZsq{}sqrt()\PYZsq{} without print math, which is \PYZsq{}math.sqrt()\PYZsq{}}
\end{Verbatim}
\end{tcolorbox}

    \begin{Verbatim}[commandchars=\\\{\}]
Square root of 16 is 4.0
    \end{Verbatim}

    \begin{tcolorbox}[breakable, size=fbox, boxrule=1pt, pad at break*=1mm,colback=cellbackground, colframe=cellborder]
\prompt{In}{incolor}{33}{\boxspacing}
\begin{Verbatim}[commandchars=\\\{\}]
\PY{c+c1}{\PYZsh{} (3) \PYZus{}\PYZus{}name\PYZus{}\PYZus{}}

\PY{k}{if} \PY{n+nv+vm}{\PYZus{}\PYZus{}name\PYZus{}\PYZus{}} \PY{o}{==} \PY{l+s+s1}{\PYZsq{}}\PY{l+s+s1}{\PYZus{}\PYZus{}main\PYZus{}\PYZus{}}\PY{l+s+s1}{\PYZsq{}}\PY{p}{:}
    \PY{n+nb}{print}\PY{p}{(}\PY{l+s+s1}{\PYZsq{}}\PY{l+s+s1}{This program is being run by itself}\PY{l+s+s1}{\PYZsq{}}\PY{p}{)}
\PY{k}{else}\PY{p}{:}
    \PY{n+nb}{print}\PY{p}{(}\PY{l+s+s1}{\PYZsq{}}\PY{l+s+s1}{I am being imported from another module}\PY{l+s+s1}{\PYZsq{}}\PY{p}{)}
\end{Verbatim}
\end{tcolorbox}

    \begin{Verbatim}[commandchars=\\\{\}]
This program is being run by itself
    \end{Verbatim}

    \begin{tcolorbox}[breakable, size=fbox, boxrule=1pt, pad at break*=1mm,colback=cellbackground, colframe=cellborder]
\prompt{In}{incolor}{34}{\boxspacing}
\begin{Verbatim}[commandchars=\\\{\}]
\PY{c+c1}{\PYZsh{} (4) build your own modules}

\PY{k}{def} \PY{n+nf}{say\PYZus{}hi}\PY{p}{(}\PY{p}{)}\PY{p}{:}
    \PY{n+nb}{print}\PY{p}{(}\PY{l+s+s1}{\PYZsq{}}\PY{l+s+s1}{Hi, this is mymodule speaking.}\PY{l+s+s1}{\PYZsq{}}\PY{p}{)}
    
\PY{n}{\PYZus{}\PYZus{}version\PYZus{}\PYZus{}} \PY{o}{=} \PY{l+s+s1}{\PYZsq{}}\PY{l+s+s1}{0.1}\PY{l+s+s1}{\PYZsq{}} \PY{c+c1}{\PYZsh{}Although it is some value inside the system, you still can change its default value.}
\end{Verbatim}
\end{tcolorbox}

    \begin{tcolorbox}[breakable, size=fbox, boxrule=1pt, pad at break*=1mm,colback=cellbackground, colframe=cellborder]
\prompt{In}{incolor}{ }{\boxspacing}
\begin{Verbatim}[commandchars=\\\{\}]

\end{Verbatim}
\end{tcolorbox}

    \begin{tcolorbox}[breakable, size=fbox, boxrule=1pt, pad at break*=1mm,colback=cellbackground, colframe=cellborder]
\prompt{In}{incolor}{ }{\boxspacing}
\begin{Verbatim}[commandchars=\\\{\}]

\end{Verbatim}
\end{tcolorbox}

    \begin{tcolorbox}[breakable, size=fbox, boxrule=1pt, pad at break*=1mm,colback=cellbackground, colframe=cellborder]
\prompt{In}{incolor}{ }{\boxspacing}
\begin{Verbatim}[commandchars=\\\{\}]

\end{Verbatim}
\end{tcolorbox}

    \begin{tcolorbox}[breakable, size=fbox, boxrule=1pt, pad at break*=1mm,colback=cellbackground, colframe=cellborder]
\prompt{In}{incolor}{ }{\boxspacing}
\begin{Verbatim}[commandchars=\\\{\}]

\end{Verbatim}
\end{tcolorbox}


    % Add a bibliography block to the postdoc
    
    
    
\end{document}
